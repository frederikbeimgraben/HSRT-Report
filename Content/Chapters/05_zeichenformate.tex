% !TEX root = ../../Main.tex
% ==============================================================================
% Kapitel 5: Zeichenformate
% ==============================================================================
% Description: Übersicht über die verwendeten Zeichenformate
% ==============================================================================

\chapter{Zeichenformate}
\label{chap:zeichenformate}

Die Zeichenformatierung ist ein wesentlicher Bestandteil der Dokumentgestaltung. Im Folgenden sind die Vorgaben für die verschiedenen Textelemente aufgeführt.

\section{Übersicht der Zeichenformate}
\label{sec:zeichenformate_uebersicht}

Die folgende Tabelle gibt eine Übersicht über alle verwendeten Zeichenformate:

\begin{table}[h]
	\centering
	\caption{Zeichenformate im Dokument}
	\label{tab:zeichenformate}
	\begin{tabular}{|l|l|c|}
		\hline
		\textbf{Element}   & \textbf{Schriftart}         & \textbf{Größe (pt)} \\
		\hline
		Titel              & Franklin Gothic Book        & 24                  \\
		Standardtext       & Franklin Gothic Book        & 12                  \\
		Überschrift 1      & Franklin Gothic Book        & 16                  \\
		Überschrift 2      & Franklin Gothic Book        & 14                  \\
		Überschrift 3      & Franklin Gothic Book (fett) & 12                  \\
		\hline
		Inhaltsverzeichnis & Franklin Gothic Book        & 10                  \\
		Abb.-verzeichnis   & Franklin Gothic Book        & 10                  \\
		Tab.-verzeichnis   & Franklin Gothic Book        & 10                  \\
		Abk.-verzeichnis   & Franklin Gothic Book        & 10                  \\
		Glossar            & Franklin Gothic Book        & 10                  \\
		\hline
		Zitate             & Franklin Gothic Book        & 12                  \\
		Kopfzeilen         & Franklin Gothic Book        & 12                  \\
		Fußzeilen          & Franklin Gothic Book        & 10                  \\
		Fußnoten           & Franklin Gothic Book        & 10                  \\
		Abstract           & Times New Roman (kursiv)    & 10                  \\
		\hline
	\end{tabular}
\end{table}

\section{Formatierungsrichtlinien}
\label{sec:formatierungsrichtlinien}

\subsection{Standardtext}
\label{subsec:standardtext}

Der Standardtext verwendet Franklin Gothic Book in 12 Punkt Größe. Diese Schriftart bietet eine gute Lesbarkeit sowohl auf Bildschirmen als auch im Druck. In \LaTeX{} wird eine passende Alternative automatisch ausgewählt, falls die exakte Schriftart nicht verfügbar ist.

\subsection{Überschriften}
\label{subsec:ueberschriften_format}

Die Hierarchie der Überschriften wird durch unterschiedliche Schriftgrößen deutlich gemacht:
\begin{listenabsatz}
	\item Überschrift 1. Ebene: 16 Punkte
	\item Überschrift 2. Ebene: 14 Punkte
	\item Überschrift 3. Ebene: 12 Punkte (fett)
\end{listenabsatz}

Alle Überschriften verwenden die gleiche Schriftart wie der Standardtext, um ein einheitliches Erscheinungsbild zu gewährleisten.

\subsection{Verzeichnisse}
\label{subsec:verzeichnisse_format}

Alle Verzeichnisse (Inhalts-, Abbildungs-, Tabellen-, Abkürzungsverzeichnis und Glossar) werden einheitlich in 10 Punkt Schriftgröße formatiert. Dies sorgt für eine kompakte Darstellung bei gleichzeitig guter Übersichtlichkeit.

\subsection{Spezielle Formatierungen}
\label{subsec:spezielle_formatierungen}

\subsubsection{Eigennamen}
Bei der Nennung von Eigennamen kann eine Kursivschrift mit Serifen verwendet werden, beispielsweise \emph{Times New Roman} in kursiv. Dies hebt Eigennamen vom restlichen Text ab.

\subsubsection{Hervorhebungen}
Auf Hervorhebungen von Text mittels Fettdruck, Unterstreichungen etc. soll generell verzichtet werden. Ausnahmen bilden:
\begin{listenabsatz}
	\item Überschriften der dritten Ebene (fett)
	\item Tabellenköpfe (fett)
	\item Eigennamen (kursiv mit Serifen)
\end{listenabsatz}

\subsubsection{Abstract}
Das Abstract auf der Titelseite verwendet als einziges Element \emph{Times New Roman} in kursiv mit 10 Punkt Schriftgröße. Diese Formatierung hebt das Abstract visuell vom restlichen Dokument ab.

\section{Tabulatoren und Abstände}
\label{sec:tabulatoren}

Der Abstand für Tabulatoren beträgt einheitlich 2\,cm. Dies gewährleistet eine konsistente Einrückung bei:
\begin{listenabsatz}
	\item Listen und Aufzählungen
	\item Eingerückten Zitaten
	\item Formeln (1\,cm Einrückung)
	\item Strukturierten Daten
\end{listenabsatz}

\section{Technische Umsetzung in \LaTeX}
\label{sec:technische_umsetzung}

In \LaTeX{} werden die Zeichenformate automatisch durch die Dokumentklasse und die gewählten Pakete umgesetzt. Die HSRTReport-Klasse definiert:
\begin{listenabsatz}
	\item Standardschriftart und -größe
	\item Überschriftformatierung durch KOMA-Script
	\item Verzeichnisformatierung durch \texttt{tocloft}
	\item Spezialformatierungen durch entsprechende Befehle
\end{listenabsatz}

Die konsistente Anwendung der Zeichenformate trägt wesentlich zur Professionalität und Lesbarkeit des Dokuments bei.

% ==============================================================================
% End of Kapitel 5
% ==============================================================================
