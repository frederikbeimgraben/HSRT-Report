%!TEX root = ../../main.tex
% ==============================================================================
% CHAPTER 1: EINLEITUNG
% ==============================================================================
\chapter{Einleitung}
\label{chap:einleitung}

% LITERATURVORSCHLÄGE FÜR KAPITEL:
% - Spektrum.de (2015): "Transkranielle Hirnstimulation als Therapie" [allgemeiner Überblick]
% - Cohen Kadosh (2015): "Enhancement of human cognitive performance using TMS" [Enhancement Kontext]
% - Violante et al. (2024): "Can neurotechnology revolutionize cognitive enhancement?" [kritische Perspektive]
% - DARPA context [militärische/praktische Anwendungen]

\section{Kontext und Relevanz}
\label{sec:kontext}

% Subsection 1.1.1: Globale Herausforderung Bildungssystem
% \subsection{Globale Herausforderungen im Bildungssystem}
% \label{subsec:globale_herausforderungen}
% - Leistungsanforderungen
% - Wettbewerb (Schulen, Universitäten, Sport)
% - Individuelle Unterschiede im Lernen
% - Alternatives zu Pharmakas?

% Subsection 1.1.2: Neurotechnologie als Lösungsansatz
% \subsection{Neurotechnologie als Lösungsansatz}
% \label{subsec:neurotechnologie}
% - Nicht-invasive Techniken
% - Neuroplastizität als Basis
% - Trendindustrie (Home-based, Consumer-Geräte)
% - Unterschied zu anderen Enhancements (Koffein, Modafinil)

\section{Forschungsfrage und Struktur}
\label{sec:forschungsfrage}

% Subsection 1.2.1: Zentrale Forschungsfrage
% \subsection{Zentrale Forschungsfrage}
% \label{subsec:zentrale_frage}
% Formulierung der Forschungsfrage
% Untergeordnete Fragen:
%   1. Wissenschaftliche Evidenz für Enhancement?
%   2. Praktische Machbarkeit?
%   3. Ethische Vertretbarkeit?

% Subsection 1.2.2: Definitionen und Abgrenzungen
% \subsection{Definitionen und Abgrenzungen}
% \label{subsec:definitionen}
% - Was ist transkranielle Hirnstimulation?
% - Enhancement vs. Therapie
% - Bildungs- vs. Trainingskontexte (unterschiedliche Populationen)
% - Fokus auf non-invasive Verfahren (tDCS, tACS, TMS)
