% !TEX root = Main.tex

% ============================================================================
% GLOSSAR: STIMULATIONSTECHNIKEN (KOMPAKT)
% ============================================================================

% tES / Transkranielle Elektrostimulation
\newglossaryentry{tES}
{
	name=Transkranielle Elektrostimulation,
	description={
			Nicht-invasives Verfahren, das schwache elektrische Ströme über Oberflächenelektroden nutzt, um neuronale Aktivität zu modulieren. Umfasst \gls{tDCS} und \gls{tACS}. \cite{woods_technical_2016}
		},
	genitive=Transkraniellen Elektrostimulation,
	plural=Transkranielle Elektrostimulationen
}

\newacronym{a:tES}{tES}{\gls{tES}}

% tDCS / Transkranielle Gleichstromstimulation
\newglossaryentry{tDCS}
{
	name=Transkranielle Gleichstromstimulation,
	description={
			Variante der \gls{tES} mit konstantem Gleichstrom (1--2\,mA, 20--30\,min). Anodale Stimulation erhöht kortikale Erregbarkeit, kathodale verringert sie. \cite{woods_technical_2016, thair_transcranial_2017}
		},
	genitive=Transkraniellen Gleichstromstimulation,
	plural=Transkranielle Gleichstromstimulationen
}

\newacronym{a:tDCS}{tDCS}{\gls{tDCS}}

% tACS / Transkranielle Wechselstromstimulation
\newglossaryentry{tACS}
{
	name=Transkranielle Wechselstromstimulation,
	description={
			Variante der \gls{tES} mit sinusförmigem Wechselstrom in verschiedenen Frequenzen. Moduliert kortikale Oszillationen durch Entrainment. Gamma (40\,Hz) und Theta (4--8\,Hz) besonders für Kognition relevant. \cite{akkad_increasing_2021}
		},
	genitive=Transkraniellen Wechselstromstimulation,
	plural=Transkranielle Wechselstromstimulationen
}

\newacronym{a:tACS}{tACS}{\gls{tACS}}

% tRNS / Transcranial Random Noise Stimulation
\newglossaryentry{tRNS}
{
	name=Transcranial Random Noise Stimulation,
	description={
			Variante der \gls{tES} mit zufällig variierenden Stromfrequenzen. Erhöht Erregbarkeit durch stochastische Resonanz, schmerzfrei. Für akademische Anwendungen (z.\,B. Mathematik).
		},
	genitive=Transkraniellen Rausch-Stimulation,
	plural=Transcranial Random Noise Stimulations
}

\newacronym{a:tRNS}{tRNS}{\gls{tRNS}}

% TMS / Transkranielle Magnetstimulation
\newglossaryentry{TMS}
{
	name=Transkranielle Magnetstimulation,
	description={
			Nicht-invasives Verfahren mit starken Magnetfeldern zur Induktion elektrischer Ströme im Gehirn (Faradaysche Induktion). Depolarisiert kortikale Neuronen. \cite{doccheck_transkranielle_nodate-1}
		},
	genitive=Transkraniellen Magnetstimulation,
	plural=Transkranielle Magnetstimulationen
}

\newacronym{a:TMS}{TMS}{\gls{TMS}}

% rTMS / Repetitive Transkranielle Magnetstimulation
\newglossaryentry{rTMS}
{
	name=Repetitive Transkranielle Magnetstimulation,
	description={
			Wiederholte \gls{TMS}-Impulse über längeren Zeitraum (Wochen) zur Induktion neuroplastischer Veränderungen. Hochfrequenz (>5\,Hz) exzitatorisch, niederfrequenz (<1\,Hz) inhibitorisch.
		},
	genitive=Repetitiven Transkraniellen Magnetstimulation,
	plural=Repetitive Transkranielle Magnetstimulationen
}

\newacronym{a:rTMS}{rTMS}{\gls{rTMS}}

% iTBS / Intermittent Theta Burst Stimulation
\newglossaryentry{iTBS}
{
	name=Intermittent Theta Burst Stimulation,
	description={
			Spezialisiertes \gls{rTMS}-Protokoll mit Bursts (3 Pulse à 50\,Hz) in Theta-Intervallen (5\,Hz) mit Pausen. Nur ~3\,min Dauer. Induziert LTP-ähnliche Effekte. \cite{huang_theta_2005, hoy_enhancement_2016}
		},
	genitive=Intermittent Theta Burst Stimulation,
	plural=Intermittent Theta Burst Stimulations
}

\newacronym{a:iTBS}{iTBS}{\gls{iTBS}}

% cTBS / Continuous Theta Burst Stimulation
\newglossaryentry{cTBS}
{
	name=Continuous Theta Burst Stimulation,
	description={
			Spezialisiertes \gls{rTMS}-Protokoll mit kontinuierlichen Theta-Bursts ohne Pausen. Induziert LTD-ähnliche, inhibitorische Effekte. \cite{huang_theta_2005}
		},
	genitive=Continuous Theta Burst Stimulation,
	plural=Continuous Theta Burst Stimulations
}

\newacronym{a:cTBS}{cTBS}{\gls{cTBS}}

% ============================================================================
% GLOSSAR: NEUROPHYSIOLOGISCHE MECHANISMEN (KOMPAKT)
% ============================================================================

% Neuroplastizität
\newglossaryentry{Neuroplasticity}
{
	name=Neuroplastizität,
	description={
			Fähigkeit des Gehirns, seine Struktur und Funktion durch Erfahrung, Lernen oder Verletzungen zu verändern. Basis für Lern- und Gedächtnisprozesse. Wird durch Hirnstimulation induziert.
		},
	genitive=Neuroplastizität,
	plural=Neuroplastizitäten
}

% LTP / Long-Term Potentiation
\newglossaryentry{LTP}
{
	name=Langzeitpotenzierung,
	description={
			Anhaltende Verstärkung synaptischer Übertragung nach hochfrequenter Stimulation. Zellulärer Mechanismus für Lernen und Gedächtnis. \gls{iTBS} induziert LTP-ähnliche Effekte. \cite{esser_level_2006, cavaleiro_memory_2020}
		},
	genitive=Langzeitpotenzierung,
	plural=Langzeitpotenzierungen
}

\newacronym{a:LTP}{LTP}{\gls{LTP}}

% LTD / Long-Term Depression
\newglossaryentry{LTD}
{
	name=Langzeitdepression,
	description={
			Anhaltende Schwächung synaptischer Übertragung nach niederfrequenter Stimulation. Mechanismus für synaptisches Pruning. \gls{cTBS} induziert LTD-ähnliche Effekte. \cite{esser_level_2006}
		},
	genitive=Langzeitdepression,
	plural=Langzeitdepressions
}

\newacronym{a:LTD}{LTD}{\gls{LTD}}

% BDNF / Brain-Derived Neurotrophic Factor
\newglossaryentry{BDNF}
{
	name=Brain-Derived Neurotrophic Factor,
	description={
			Neurotropher Wachstumsfaktor, der synaptische Plastizität und Synapsenbildung fördert. Wird durch Lernprozesse und Hirnstimulation (besonders \gls{iTBS}) hochreguliert. BDNF-Polymorphismen können Responsiveness vorhersagen. \cite{cavaleiro_memory_2020}
		},
	genitive=Brain-Derived Neurotrophic Factor,
	plural=Brain-Derived Neurotrophic Factors
}

\newacronym{a:BDNF}{BDNF}{\gls{BDNF}}

% Entrainment
\newglossaryentry{Entrainment}
{
	name=Entrainment,
	description={
			Synchronisation neuronaler Oszillationen mit externem rhythmischem Input (\gls{tACS}-Frequenz). Ermöglicht spezifische Hirnrhythmen-Modulation. Hypothetischer Mechanismus für frequenzabhängige Effekte. \cite{akkad_increasing_2021}
		},
	genitive=Entrainment,
	plural=Entrainments
}

% ============================================================================
% GLOSSAR: TECHNISCHE PARAMETER (KOMPAKT)
% ============================================================================

% Stromstärke und Stromdichte
\newglossaryentry{CurrentIntensity}
{
	name=Stromstärke und Stromdichte,
	description={
			Stromstärke in mA (\gls{tDCS}: 1--2\,mA). Stromdichte in mA/cm² besser geeignet zur Vergleichbarkeit. Typisch: 0,029--0,08\,mA/cm² für kognitive Anwendungen. \cite{woods_technical_2016}
		},
	genitive=Stromstärke und Stromdichte,
	plural=Stromstärken und Stromdichten
}

% Anode und Kathode
\newglossaryentry{Electrode}
{
	name=Anode und Kathode,
	description={
			Bei \gls{tDCS}: Anode (positive Elektrode) erhöht Erregbarkeit, Kathode (negative) verringert sie. Effektrichtung durch Ruhemembranpotenzial-Verschiebung bestimmt. \cite{woods_technical_2016}
		},
	genitive=Anode und Kathode,
	plural=Anoden und Kathoden
}

% Elektrodenmontage und 10-20-System
\newglossaryentry{Montage}
{
	name=Elektrodenmontage und 10-20-System,
	description={
			Elektrodenmontage beschreibt räumliche Anordnung auf Kopfhaut nach 10-20-EEG-Standard (F=frontal, C=central, P=parietal, O=occipital; ungerade=links, gerade=rechts). Z.\,B. F3 Anode über DLPFC. 1\,cm Verschiebung ändert Stromverteilung signifikant. \cite{woods_technical_2016, caulfield_optimized_2022}
		},
	genitive=Elektrodenmontage und 10-20-System,
	plural=Elektrodenmontagen und 10-20-Systeme
}

% Neuronavigation
\newglossaryentry{Neuronavigation}
{
	name=Neuronavigation,
	description={
			Verwendung individueller MRT/fMRI zur präzisen Elektrodenplatzierung und Stromflussvorhersage. Verbessert Targeting-Genauigkeit über Standard-Systemen. Oft gekoppelt mit Finite-Elemente-Modellen. \cite{woods_technical_2016, meinzer_investigating_2024}
		},
	genitive=Neuronavigation,
	plural=Neuronavigationen
}

% Sham-Stimulation
\newglossaryentry{ShamStimulation}
{
	name=Sham-Stimulation,
	description={
			Placebo-Kontrollbedingung in Studien. Bei \gls{tDCS}: kurze Stromapplikation am Anfang (z.\,B. 30\,s mit Ramp) für realistische Hautempfindungen, ohne neurobiologische Effekte. Ermöglicht blinde Studiendesigns. \cite{woods_technical_2016}
		},
	genitive=Sham-Stimulation,
	plural=Sham-Stimulationen
}

% ============================================================================
% GLOSSAR: KOGNITIVE DOMÄNEN (KOMPAKT)
% ============================================================================

% Arbeitsgedächtnis
\newglossaryentry{WorkingMemory}
{
	name=Arbeitsgedächtnis,
	description={
			Kognitives System für temporäre Speicherung und Manipulation von Informationen. Primär im \gls{DLPFC} verarbeitet. Limitiert (~4 Items). Typisch gemessen mit n-back-Aufgaben. Hirnstimulation verbessert moderat. \cite{hoy_enhancement_2016, senkowski_boosting_2022}
		},
	genitive=Arbeitsgedächtnis,
	plural=Arbeitsgedächtnisse
}

% Gedächtniskonsolidierung
\newglossaryentry{MemoryConsolidation}
{
	name=Gedächtniskonsolidierung,
	description={
			Prozess der Stabilisierung von Gedächtnisinhalten nach Erwerb. Zwei Phasen: Intra-Session (Minuten) und Inter-Session (Stunden--Wochen). Schlaf (v.\,a. Tiefschlaf) kritisch. Hirnstimulation kann Konsolidierung beschleunigen. \cite{reis_noninvasive_2009}
		},
	genitive=Gedächtniskonsolidierung,
	plural=Gedächtniskonsolidierungen
}

% Motorisches Lernen
\newglossaryentry{MotorLearning}
{
	name=Motorisches Lernen,
	description={
			Erwerb motorischer Fähigkeiten durch Übung. Online-Lernen (während Training) und Offline-Konsolidierung (danach). \gls{M1} kritisch für Konsolidierung. Anodale \gls{tDCS} über M1 verbessert Konsolidierung. \cite{reis_noninvasive_2009}
		},
	genitive=Motorischen Lernen,
	plural=Motorische Lernens
}

% ============================================================================
% GLOSSAR: HIRNREGIONEN (KOMPAKT)
% ============================================================================

% DLPFC / Dorsolateral Prefrontal Cortex
\newglossaryentry{DLPFC}
{
	name=Dorsolateraler Präfrontaler Kortex,
	description={
			Hirnregion im Stirnlappen. Zentral für Arbeitsgedächtnis, exekutive Funktionen, kognitive Kontrolle. Häufigstes Stimulationsziel für kognitive Verbesserung. Im 10-20-System: F3 (links) oder F4 (rechts). \cite{hoy_enhancement_2016, woods_technical_2016}
		},
	genitive=Dorsolateralen Präfrontalen Kortex,
	plural=Dorsolaterale Präfrontale Kortizes
}

\newacronym{a:DLPFC}{DLPFC}{\gls{DLPFC}}

% M1 / Primary Motor Cortex
\newglossaryentry{M1}
{
	name=Primärer Motorischer Kortex,
	description={
			Hirnregion für willkürliche Bewegungssteuerung. Kritisch für motorisches Lernen und Fertigkeitserwerb. Häufigstes Stimulationsziel bei motorischen Anwendungen. Im 10-20-System: C3 (links) oder C4 (rechts). Anodale \gls{tDCS} verbessert Konsolidierung. \cite{reis_noninvasive_2009}
		},
	genitive=Primären Motorischen Kortex,
	plural=Primäre Motorische Kortizes
}

\iffalse
	% ============================================================================
	% GLOSSAR: VARIABILITÄT UND METHODIK (KOMPAKT)
	% ============================================================================

	% Inter-individuelle Variabilität
	\newglossaryentry{InterIndividVariability}
	{
		name=Inter-individuelle Variabilität,
		description={
				Unterschiede zwischen Individuen in Responsiveness gegenüber Hirnstimulation. Problem: nur 39--45\% zeigen erwartete Effekte. Ursachen: anatomische Unterschiede (Kortexdicke, Gyrierung), genetische Faktoren (\gls{BDNF}-Polymorphismen), funktionelle Baseline-Unterschiede. \cite{vergallito_inter-individual_2022, chew_inter-_2015}
			},
		genitive=Inter-individuellen Variabilität,
		plural=Inter-individuellen Variabilitäten
	}

	% Responder / Non-Responder
	\newglossaryentry{Responder}
	{
		name=Responder und Non-Responder,
		description={
				Responder: Personen mit erwarteten Verbesserungen nach Hirnstimulation. Non-Responder: keine signifikanten Effekte. Quote: 39--45\% Responder. Identifikation und Personalisierung = aktives Forschungsfeld. \cite{vergallito_inter-individual_2022}
			},
		genitive=Responder und Non-Responder,
		plural=Responder und Non-Responder
	}

	% Meta-Analyse
	\newglossaryentry{MetaAnalysis}
	{
		name=Meta-Analyse,
		description={
				Statistische Synthese von Ergebnissen mehrerer Studien. Berechnet gepoolte Effektgrößen (\gls{SMD}) und Konfidenzintervalle. Standard für Wirksamkeits-Einschätzung. \cite{senkowski_boosting_2022, simonsmeier_electrical_2018}
			},
		genitive=Meta-Analyse,
		plural=Meta-Analysen
	}

	% SMD / Standardized Mean Difference
	\newglossaryentry{SMD}
	{
		name=Standardized Mean Difference,
		description={
				Effektstärkemaß in Meta-Analysen (Cohen's d). Differenz zwischen Gruppen geteilt durch gepoolte Standardabweichung. Interpretation: 0,2=klein, 0,5=mittel, 0,8=groß. Ermöglicht Vergleiche über verschiedene Messungen. \cite{senkowski_boosting_2022}
			},
		genitive=Standardized Mean Difference,
		plural=Standardized Mean Differences
	}

	\newacronym{a:SMD}{SMD}{\gls{SMD}}

	% RCT / Randomized Controlled Trial
	\newglossaryentry{RCT}
	{
		name=Randomized Controlled Trial,
		description={
				Randomisiert kontrollierte Studie. Gold-Standard für Kausalitäts-Nachweis. Probanden zufällig in Interventions- und Kontrollgruppen eingeteilt. Standard zur Validierung von Stimulations-Protokollen.
			},
		genitive=Randomized Controlled Trial,
		plural=Randomized Controlled Trials
	}

	\newacronym{a:RCT}{RCT}{\gls{RCT}}
\fi
