% !TEX root = ../../Main.tex

% Meeting date
\newcommand{\waterMarkText}{}

% Current date and time
\createdon{\today}

% Title
\title{Ist transkranielle Hirnstimulation eine praktikable Methode zur Beschleunigung des Fertigkeitserwerbs im Bildungskontext?}

% Data fields for the title page
\makeatletter
\AddTitlePageDataLine{Thema}{Thema-038:}
\AddTitlePageDataLine{}{Transkranielle Hirnstimulation zur Förderung des Denkens und des Lernens}
\AddTitlePageDataSpace{3pt}
% Deutsch
\AddTitlePageDataLine{Keywords (Deutsch)}{Transkranielle Hirnstimulation, Neuroenhancement, Fertigkeitserwerb}
\AddTitlePageDataLine{}{Bildung und Training, Neuroplastizität}
% Englisch
\AddTitlePageDataLine{Keywords (English)}{Transcranial Stimulation, Neuroenhancement, Skill Acquisition}
\AddTitlePageDataLine{}{Education and Training, Neuroplasticity}
\AddTitlePageDataSpace{5pt}
\AddTitlePageDataLine{Wortanzahl}{\quickwordcount{Content/01_content}} % !!! Only 01_content.tex !!!
\AddTitlePageDataSpace{5pt}
\AddTitlePageDataLine{Vorgelegt von}{Frederik Beimgraben}
\AddTitlePageDataLine{}{6. Fachsemester}
\AddTitlePageDataLine{}{\href{mailto:frederik.beimgraben@student.reutlingen-university.de}{frederik.beimgraben@student.reutlingen-university.de}}
\AddTitlePageDataSpace{5pt}
\AddTitlePageDataLine{Vorgelegt am}{\today}
\AddTitlePageDataSpace{5pt}
\AddTitlePageDataLine{Studiengang}{Medizinisch Technische Informatik B.Sc.}
\AddTitlePageDataLine{Professor:in}{Prof. Dr. Sven Steddin}
\AddTitlePageDataLine{Semester}{Wintersemester 2025/2026}
\AddTitlePageDataLine{Modul}{METI6.3}
\AddTitlePageDataLine{}{Seminar Ausgewählter Themen der Medizinisch-Technischen Informatik}
\makeatother

% Abstract
\newcommand{\titlepageabstract}{%
	Das Abstract beschreibt in wenigen Sätzen die Zielsetzung und das Ergebnis der Ausarbeitung. Das Abstract muss sich vollständig auf der Titelseite befinden. Die Zeichensatzformatierung wird in einem eigenen Absatz beschrieben  Das Abstract soll es den Lesern:innen ermöglichen, innerhalb von wenigen Augenblicken zu erfassen, welcher Inhalt hinter der Überschrift steckt und ob das Thema, aus Sicht der Leser:innen, zur weiteren Bearbeitung lohnt. Das Abstract ist keine verbale Beschreibung des Inhaltsverzeichnisses, sondern gibt kurz und knapp z.B. die Zielsetzung (z.B. Hypothese), die eingesetzten Methoden und die erzielten Ergebnisse / Erkenntnisse bekannt. Weitere Hinweise finden Sie außerdem im Vorlesungsskript.
}

% Disable indentation
\setlength{\parindent}{0pt}
