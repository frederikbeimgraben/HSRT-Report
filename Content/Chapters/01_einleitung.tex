% !TEX root = ../../Main.tex
% ==============================================================================
% Kapitel 1: Einleitung
% ==============================================================================
% Description: Einleitung zur Formatierung der Seminararbeit
% ==============================================================================

\chapter{Einleitung}
\label{chap:einleitung}

Das hier vorliegende Dokument soll als Vorlage für die Gestaltung der im Seminarkurs \emph{Ausgewählte Themen der Medizinisch-Technischen Informatik} erstellten schriftlichen Ausarbeitungen dienen.

Es ist zu beachten, dass das Formatieren des Textes als einer der letzten Schritte der Ausarbeitung durchgeführt wird, da dieser Schritt erfahrungsgemäß viel Zeit in Anspruch nimmt und daher nur einmalig ausgeführt werden sollte.

Der vorliegende Entwurf wurde mit \LaTeX{} unter Verwendung der HSRTReport-Dokumentklasse erstellt, welche auf der KOMA-Script-Klasse \texttt{scrreprt} basiert. Diese Vorlage bietet eine professionelle und konsistente Formatierung für wissenschaftliche Arbeiten.

Alle in diesem Dokument enthaltenen Hinweise zur Gestaltung des Dokumentes dienen als Referenz für die korrekte Verwendung des Templates. Die Vorlage selbst übernimmt bereits die meisten Formatierungsanforderungen automatisch.

% ==============================================================================
% End of Kapitel 1
% ==============================================================================
