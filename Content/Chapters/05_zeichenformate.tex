% !TEX root = ../../Main.tex
% ==============================================================================
% Kapitel 5: Zeichenformate
% ==============================================================================
% Description: Übersicht über die verwendeten Zeichenformate
% ==============================================================================

\chapter{Zeichenformate}
\label{chap:zeichenformate}

Die Zeichenformatierung ist ein wesentlicher Bestandteil der Dokumentgestaltung. Im Folgenden sind die Vorgaben für die verschiedenen Textelemente aufgeführt.

\section{Übersicht der Zeichenformate}
\label{sec:zeichenformate_uebersicht}

Die folgende Tabelle gibt eine Übersicht über alle verwendeten Zeichenformate:

\begin{table}[h]
	\centering
	\caption{Zeichenformate im Dokument}
	\label{tab:zeichenformate}
	\begin{tabular}{|l|l|c|}
		\hline
		\textbf{Element}              & \textbf{Schriftart} & \textbf{Größe}       \\
		\hline
		Titel                         & Blender-Bold        & 24 pt                \\
		Standardtext                  & DIN-Regular         & 11 pt                \\
		Kapitel (Chapter)             & Blender-Bold        & \textbackslash LARGE \\
		Überschrift 1 (Section)       & Blender-Bold        & \textbackslash Large \\
		Überschrift 2 (Subsection)    & Blender-Bold        & \textbackslash large \\
		Überschrift 3 (Subsubsection) & Blender-Bold        & \textbackslash large \\
		\hline
		Inhaltsverzeichnis            & Blender-Bold        & 10                   \\
		Abb.-verzeichnis              & Blender-Bold        & 10                   \\
		Tab.-verzeichnis              & Blender-Bold        & 10                   \\
		Abk.-verzeichnis              & Blender-Bold        & 10                   \\
		Glossar                       & Blender-Bold        & 10                   \\
		\hline
		Zitate                        & DIN-Regular         & 12                   \\
		Kopfzeilen                    & Blender-Bold        & 10                   \\
		Fußzeilen                     & Blender-Bold        & 10                   \\
		Fußnoten                      & DIN-Regular         & 9                    \\
		Abstract                      & DIN-Regular         & 11                   \\
		\hline
	\end{tabular}
\end{table}

\section{Formatierungsrichtlinien}
\label{sec:formatierungsrichtlinien}

\subsection*{Standardtext}
\label{subsec:standardtext}

Der Standardtext verwendet Franklin Gothic Book (oder die in \LaTeX{} konfigurierte Alternative \texttt{blenderfont}) in 11 Punkt Größe als Basisschriftgröße. Der Zeilenabstand ist auf 1,5-fach eingestellt (\texttt{baselinestretch=1.5}), was eine gute Lesbarkeit sowohl auf Bildschirmen als auch im Druck gewährleistet.

\subsection*{Überschriften}
\label{subsec:ueberschriften_format}

Die Hierarchie der Überschriften wird durch unterschiedliche \LaTeX{}-Schriftgrößenbefehle und vertikale Abstände deutlich gemacht:
\begin{listenabsatz}
	\item Kapitel (Chapter): \textbackslash LARGE mit 3ex Abstand davor
	\item Überschrift 1. Ebene (Section): \textbackslash Large mit 4,5ex Abstand davor
	\item Überschrift 2. Ebene (Subsection): \textbackslash large mit 3,5ex Abstand davor
	\item Überschrift 3. Ebene (Subsubsection): \textbackslash large mit 2ex Abstand davor
\end{listenabsatz}

Die großzügigen vertikalen Abstände vor Sections und Subsections sorgen für eine klare visuelle Gliederung des Dokuments. Alle Überschriften verwenden Fettdruck (\texttt{\textbackslash bfseries}) zusätzlich zur \texttt{blenderfont}.

Alle Überschriften verwenden die gleiche Schriftart wie der Standardtext (\texttt{blenderfont}), jedoch mit Fettdruck versehen, um ein einheitliches und hierarchisches Erscheinungsbild zu gewährleisten.

\subsection*{Verzeichnisse}
\label{subsec:verzeichnisse_format}

Alle Verzeichnisse (Inhalts-, Abbildungs-, Tabellen-, Abkürzungsverzeichnis und Glossar) werden einheitlich in 10 Punkt Schriftgröße formatiert. Dies sorgt für eine kompakte Darstellung bei gleichzeitig guter Übersichtlichkeit.

\subsection*{Spezielle Formatierungen}
\label{subsec:spezielle_formatierungen}

\subsubsection{Eigennamen}
Bei der Nennung von Eigennamen kann eine Kursivschrift mit Serifen verwendet werden, beispielsweise \emph{Times New Roman} in kursiv. Dies hebt Eigennamen vom restlichen Text ab.

\subsubsection{Hervorhebungen}
Auf Hervorhebungen von Text mittels Fettdruck, Unterstreichungen etc. soll generell verzichtet werden. Ausnahmen bilden:
\begin{listenabsatz}
	\item Überschriften der dritten Ebene (fett)
	\item Tabellenköpfe (fett)
	\item Eigennamen (kursiv mit Serifen)
\end{listenabsatz}

\subsubsection{Abstract}
Das Abstract auf der Titelseite verwendet als einziges Element \emph{Times New Roman} in kursiv mit 10 Punkt Schriftgröße. Diese Formatierung hebt das Abstract visuell vom restlichen Dokument ab.

\section{Tabulatoren und Abstände}
\label{sec:tabulatoren}

Der Abstand für Tabulatoren beträgt einheitlich 2\,cm. Dies gewährleistet eine konsistente Einrückung bei:
\begin{listenabsatz}
	\item Listen und Aufzählungen
	\item Eingerückten Zitaten
	\item Formeln (1\,cm Einrückung)
	\item Strukturierten Daten
\end{listenabsatz}

\section{Technische Umsetzung in \LaTeX}
\label{sec:technische_umsetzung}

In \LaTeX{} werden die Zeichenformate automatisch durch die Dokumentklasse und die gewählten Pakete umgesetzt. Die HSRTReport-Klasse definiert:
\begin{listenabsatz}
	\item Basisschriftgröße: 11pt (durch Dokumentklassen-Option)
	\item Schriftart: \texttt{blenderfont} (Franklin Gothic Book oder Systemalternative)
	\item Überschriftformatierung durch KOMA-Script mit erweiterten Abständen
	\item Verzeichnisformatierung durch \texttt{tocloft} mit \texttt{blenderfont}
	\item Automatische Seitenumbruchkontrolle durch \texttt{PageBreakControl.tex}
	\item Zeilenabstand: 1,5-fach (\texttt{baselinestretch=1.5})
	\item Absatzabstand: 6pt (\texttt{parskip=6pt})
\end{listenabsatz}

Die konsistente Anwendung der Zeichenformate trägt wesentlich zur Professionalität und Lesbarkeit des Dokuments bei.

% ==============================================================================
% End of Kapitel 5
% ==============================================================================
