% !TEX root = ../../Main.tex
% ==============================================================================
% Kapitel 2: Gliederung des Textes
% ==============================================================================
% Description: Anforderungen an die Gliederung der Seminararbeit
% ==============================================================================

\chapter{Gliederung des Textes}
\label{chap:gliederung}

Nach jedem der nachfolgend genannten Abschnitte muss mit einer neuen Seite begonnen werden. Die Ausarbeitung muss die nachfolgend gegebene Gliederung aufweisen (Hauptüberschriften).

Die als Liste angegeben Stichpunkte beschreiben, welche Inhalte in den Abschnitten behandelt werden sollten. Diese Stichpunkte sind, mit Ausnahme des Abschnitts „Verzeichnisse", nicht zwingend als Unterüberschriften vorgegeben, können aber in gleicher oder ähnlicher Form verwendet werden, sofern dies sinnvoll erscheint. Achten Sie hierbei insbesondere darauf, dass ein mit einer Überschrift versehener Textblock nicht nur aus einem oder wenigen Sätzen bestehen darf.

\section{Erforderliche Gliederung}
\label{sec:erforderliche_gliederung}

Die Seminararbeit muss folgende Struktur aufweisen:

\subsection{Titelseite mit Abstract}
\label{subsec:titelseite}

Die Titelseite enthält alle wesentlichen Informationen zur Arbeit sowie das Abstract und die Keywords.

\subsection{Verzeichnisse}
\label{subsec:verzeichnisse}

Folgende Verzeichnisse sind zu erstellen:
\begin{listenabsatz}
	\item Inhaltsverzeichnis
	\item Abbildungsverzeichnis
	\item Tabellenverzeichnis
	\item Formelverzeichnis
	\item Abkürzungsverzeichnis
	\item Glossar
\end{listenabsatz}

\subsection{Einleitung}
\label{subsec:einleitung_struktur}

Die Einleitung sollte folgende Punkte behandeln:
\begin{listenabsatz}
	\item Beschreibung des Problems und der Forschungsaufgabe
	\item Verdeutlichung der Relevanz für Wissenschaft und Gesellschaft
	\item Hypothese
	\item Definition der Leitfragen
	\item Stand der Wissenschaft/Technik
	\item Vorgehensweise zur Verifikation der Hypothese
\end{listenabsatz}

\subsection{Methoden}
\label{subsec:methoden_struktur}

Der Methodenteil umfasst:
\begin{listenabsatz}
	\item Beschreibung der Umsetzung der Vorgehensweise zur Erzielung der gesuchten Ergebnisse (z.\,B. Aufbau der Messtechnik und Ablauf der Experimente, Beschreibung der Arbeitsinstrumente und Werkzeuge, Beschreibung der Lösungsprozesse oder Vorgehensweise bei der Literaturrecherche/Selektionskriterien)
	\item Methoden müssen so beschrieben sein, dass andere Personen das Verfahren nachvollziehen/reproduzieren können
\end{listenabsatz}

\subsection{Ergebnisse}
\label{subsec:ergebnisse_struktur}

Im Ergebnisteil erfolgt:
\begin{listenabsatz}
	\item Darstellung der über die Untersuchungsmethoden erzielten Ergebnisse (objektive Darstellung)
	\item Stellungnahme zur Verifikation der Hypothese durch Beantwortung der Leitfragen
\end{listenabsatz}

\subsection{Diskussion}
\label{subsec:diskussion_struktur}

Die Diskussion beinhaltet:
\begin{listenabsatz}
	\item Interpretation der Ergebnisse (was lässt sich aus den Daten folgern -- subjektive Beurteilung/persönliche Meinung)
	\item Vergleich der Ergebnisse mit den bisher bekannten Daten (Bewertung der Ergebnisse aus der Literatur: gibt es Übereinstimmung oder Widersprüche; wie lassen sich mögliche Widersprüche erklären?)
\end{listenabsatz}

\subsection{Zusammenfassung}
\label{subsec:zusammenfassung_struktur}

Die Zusammenfassung enthält:
\begin{listenabsatz}
	\item Kurze Beschreibung der Fragestellung und der Ergebnisse
	\item Ausblick und Empfehlungen
\end{listenabsatz}

\subsection{Literaturverzeichnis}
\label{subsec:literatur_struktur}

Das Literaturverzeichnis listet alle verwendeten Quellen auf.

\subsection{Danksagungen}
\label{subsec:danksagungen_struktur}

Optional können Danksagungen eingefügt werden:
\begin{listenabsatz}
	\item Benennung der Sponsoren
	\item Nennung der Hilfspersonen und deren Aufgabengebiet
\end{listenabsatz}

\subsection{Eidesstattliche Erklärung}
\label{subsec:erklaerung_struktur}

Die eidesstattliche Erklärung muss unterschrieben werden (z.\,B. durch die Verwendung der PDF-Unterschriftsfunktion im Acrobat Reader).

\subsection{Anhang}
\label{subsec:anhang_struktur}

Der Anhang kann enthalten:
\begin{listenabsatz}
	\item Arbeitshypothese
	\item Leitfragen mit Validierung
	\item Projektplan (Gantt-Diagramm)
	\item Verbesserungsvorschläge (nur in Version 2)
\end{listenabsatz}

% ==============================================================================
% End of Kapitel 2
% ==============================================================================
