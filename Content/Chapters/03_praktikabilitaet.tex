%!TEX root = ../../main.tex
% ==============================================================================
% CHAPTER 3: PRAKTIKABILITÄT
% ==============================================================================
\chapter{Praktikabilität in Bildungs- und Trainingskontexten}
\label{chap:praktikabilität}

% LITERATURVORSCHLÄGE FÜR KAPITEL:
% - Cappon et al. (2023): "Home-based tES training program" [Machbarkeit, 10 Zit.]
% - Woodham et al. (2024): "Home-based tDCS in MDD" [Nature Medicine, 62 Zit., praktisches Modell]
% - Woods et al. (2015/2016): "Technical guide to tDCS" [Standardwerk, 1700+ Zit.]
% - Thair et al. (2017): "Transcranial Direct Current Stimulation (tDCS)" [Praktische Details]
% - Caulfield et al. (2022): "Optimized electrode positions" [Technische Optimierung]

\section{Technische Aspekte und Durchführbarkeit}
\label{sec:tech_durchführbarkeit}

% LITERATUR: Woods 2015/2016, Thair 2017, Caulfield 2022

% \subsection{Kosteneffizienz und ökonomische Zugänglichkeit}
\label{subsec:kosten}

% LITERATUR: neurocare academy, Woodham 2024, Cappon 2023

% Subsubsection 3.1.1.1: Gerätekosten
% \subsubsection{Gerätekosten und Verfügbarkeit}
\label{subsubsec:gerätekosten}
% - tDCS Consumer-Geräte: $200-500
% - rTMS klinische Geräte: $10,000-50,000
% - tACS/tRNS: mittleres Preissegment
% - Vergleich:
%   - Private Nachhilfe/Einzelunterricht: €30-60/h
%   - Elite-Trainingsprogramme: €1000+/Monat
%   - tDCS: ~€5-10 pro Sitzung (Amortisation)

% Subsubsection 3.1.1.2: Kostenvergleich zu Alternativen
% \subsubsection{Kostenvergleich zu alternativen Interventionen}
\label{subsubsec:kostenvergleich}
% - Vs. Einzelunterricht: tDCS kostengünstiger bei Skalierung
% - Vs. Medikamente (Modafinil, etc.): vergleichbar
% - Vs. kognitive Trainings-Apps: tDCS höher initial, aber kombinierbar

% \subsection{Heimbasierte und dezentrale Anwendung}
\label{subsec:heimbasiert}

% LITERATUR: Cappon 2023 (10 Zit., Pilotprogramm), Woodham 2024 (Nature Med, 62 Zit., 174 Probanden)

% Subsubsection 3.1.2.1: Erfolgreiche Pilotprojekte
% \subsubsection{Erfolgreiche Pilotprojekte mit Heimanwendung}
\label{subsubsec:pilotprojekte}
% - Cappon et al. (2023): Home-based tES Trainings-Programm
%   - Laien-Administration nach Trainig
%   - Adhärenz: 98-100%
%   - Sicherheit demonstriert
%   - Downloadbare Materialien verfügbar
% - Woodham et al. (2024, Nature Medicine): Home-based tDCS bei MDD
%   - 174 Teilnehmer, remote supervision
%   - 10 Wochen Protokoll
%   - Hohe Akzeptanz, gute Effektivität

% Subsubsection 3.1.2.2: Remote Supervision und Monitoring
% \subsubsection{Remote Supervision und Monitoring}
\label{subsubsec:remote_supervision}
% - Möglichkeit mit modernen Technologien
% - Telemedizin-Modelle existieren
% - Dokumentation via Apps/Cloud
% - Sicherheit: Contact person im Notfall nötig

% Subsubsection 3.1.2.3: Skalierbarkeit für breite Populationen
% \subsubsection{Skalierbarkeit für breite Populationen}
\label{subsubsec:skalierbarkeit}
% - Implication: Theoretisch skalierbar
% - Unterschied:
%   - Élite-Training (kleine Gruppen, hohe Personalkosten): möglich
%   - Mainstream-Bildung (Millionen Schüler): logistisch schwierig
%   - Hybrid-Modell: vielversprechend (Schulungen + Home-Use)

% \subsection{Sicherheit und Nebenwirkungsprofil}
\label{subsec:sicherheit}

% LITERATUR: Woods 2016 (technischer Standard), Wassermann-Kriterien, McCambridge et al. Sicherheit-Reviews

% Subsubsection 3.1.3.1: Häufige milde Nebenwirkungen
% \subsubsection{Häufige und milde Nebenwirkungen}
\label{subsubsec:mild_nebenwirkungen}
% - Kopfschmerzen: 10-20% (mit rezeptfreien Schmerzmitteln behandelbar)
% - Kopfhaut-Irritation/Brennen: lokal, vorübergehend
% - Muskelzuckungen (bei TMS): während Stimulation, gewöhnung nach wenigen Sessions
% - Müdigkeit/Benommenheit: selten, vorübergehend (<1h)
% - Übelkeit: sehr selten

% Subsubsection 3.1.3.2: Schwerwiegende Komplikationen
% \subsubsection{Schwerwiegende Komplikationen}
\label{subsubsec:severe_complications}
% - Epileptische Anfälle: extrem selten (<0.1% bei compliance mit Wassermann-Kriterien)
% - Keine bekannten Fälle permanenter Hirnschädigung
% - Vergleich zu Medikamenten: günstiger Effekt-Nebenwirkung-Profile
% - Kontraindikationen gut etabliert

% Subsubsection 3.1.3.3: Vergleich zu pharmakologischen Enhancern
% \subsubsection{Vergleich zu pharmakologischen Enhancern}
\label{subsubsec:pharm_vs_stim}
% - tDCS: keine systemischen Effekte (lokal wirksam)
% - Modafinil/Amphetamine: systemische Nebeneffekte (kardiovaskular, GI, etc.)
% - Koffein: Entzugssymptome, Toleranz
% - Fazit: tDCS sicherere Alternative aus Nebenwirkungs-Perspektive

\section{Praktische Implementierungshürden}
\label{sec:hürden}

% LITERATURVORSCHLÄGE: Vergallito 2022, Woods 2016, Meinzer 2024, Cappon 2023

% \subsection{Personalisierung und Vorhersage von Responsiveness}
\label{subsec:personalisierung}

% LITERATUR: Vergallito 2022, Meinzer 2024 (neuroimaging context), computational modeling papers

% Subsubsection 3.2.1.1: Das Problem: 60\% Non-Responder
% \subsubsection{Das Problem der 60\% Non-Responder}
\label{subsubsec:nonresponder_problem}
% - Vergallito (2022): Nur 39-45% zeigen erwartete Effekte
% - Current Standardprotokolle: "One-Size-Fits-All" funktioniert nicht
% - Screening nötig ODER Individualisierung VOR Behandlung

% Subsubsection 3.2.1.2: Anatomische Personalisierung
% \subsubsection{Anatomische Personalisierung}
\label{subsubsec:anat_personalisierung}
% - MRT-basierte Modelle: Individuelle Elektrodenplatzierung
% - Finite-Elemente-Modelle (FEM): Stromfluss berechnen
% - Toolboxen: SimNIBS, COMETS (MATLAB)
% - Kosten/Nutzen: Trade-off zwischen Präzision und praktische Durchführbarkeit

% Subsubsection 3.2.1.3: Funktionelle und genetische Personalisierung
% \subsubsection{Funktionelle und genetische Biomarker}
\label{subsubsec:biomarker_personal}
% - Theta-Power im Frontallappen (prädiziert rTMS-Response)
% - BDNF-Polymorphismen, COMT, DAT
% - EEG-basierte adaptive Protokolle (Closed-Loop)
% - Forschungsbedarf: Validierung dieser Biomarker

% \subsection{Standardisierung und Schulung von Anwendern}
\label{subsec:standardisierung}

% LITERATUR: Cappon 2023 (Trainings-Programm), neurocare academy, IFCN Guidelines

% Subsubsection 3.2.2.1: Notwendigkeit von Standardisierten Protokollen
% \subsubsection{Standardisierte Protokolle und Guidelines}
\label{subsubsec:standardisierte_protokolle}
% - Unterschiede aktuelle Praxis: große Variabilität
% - IFCN Guidelines (International Federation of Clinical Neurophysiology)
% - Wassermann-Kriterien für Sicherheit
% - Notwendig: Einigung auf "Best Practice" Protokolle

% Subsubsection 3.2.2.2: Trainingsmaterialien für nicht-medizinisches Personal
% \subsubsection{Trainingsmaterialien für Laien}
\label{subsubsec:training_laien}
% - Cappon (2023) hat Programm entwickelt: Video, Quizze, Checklisten
% - Erfolgreiche Durchführbarkeit demonstriert
% - Unterschied: Laien können Geräte anlegen, aber müssen medizinisch überwacht werden

% Subsubsection 3.2.2.3: Regulatorische Fragen zur Anwender-Qualifikation
% \subsubsection{Regulatorische Fragen}
\label{subsubsec:regulatory_qual}
% - Wer darf tDCS/TMS anwenden?
% - Ärzte, Therapeuten, Trainer, Laien?
% - Versicherungsdeckung, Haftung
% - Zertifizierungspfade nötig

% \subsection{Realistische Effektgrößen und Erwartungsmanagement}
\label{subsec:effektgrößen}

% LITERATUR: Reis 2009, Akkad 2021, Simonsmeier 2018, Violante 2024

% Subsubsection 3.2.3.1: Evidenzbasierte Effektgrößen
% \subsubsection{Empirische Effektgrößen aus Studien}
\label{subsubsec:empirisch_effekt}
% - Motorisches Lernen: Beschleunigung ~10-30% [Reis 2009, Akkad 2021]
% - Akademische Fähigkeiten: SMD = 0.3-0.7 [Simonsmeier 2018]
% - Arbeitsgedächtnis (Multi-Session): SMD ~0.3-0.5
% - NICHT: Keine "Wunder-Effekte"

% Subsubsection 3.2.3.2: Vergleich zu anderen Lerntechniken
% \subsubsection{Benchmarking gegen etablierte Methoden}
\label{subsubsec:benchmarking}
% - Spaced Repetition: SMD ~0.5-1.0
% - Elaboration: SMD ~0.7
% - tDCS: SMD ~0.3-0.7 (abhängig von Domäne)
% - Fazit: Vergleichbar zu anderen Techniken, nicht überlegen

% Subsubsection 3.2.3.3: "Nicht-Ersatz für gutes Lehren"
% \subsubsection{tDCS ersetzt nicht gutes Unterrichten}
\label{subsubsec:not_replacement}
% - tDCS ist ZUSATZ zu optimiertem Unterricht
% - Basissystem muss gut sein (sonst "garbage in, garbage out")
% - Kombination wichtig: tDCS + guter Trainer/Lehrer + strukturiertes Programm

\section{Praktikabilitäts-Fazit: Realistisches Szenario}
\label{sec:praktik_fazit}

% LITERATUR: Violante 2024 (kritikal), Cappon/Woodham für Machbarkeit

% \subsection{Machbarkeit nach Setting}
\label{subsec:machbarkeit_setting}

% Subsubsection 3.3.1.1: Élite-Training und spezialisierte Kontexte
% \subsubsection{Élite-Training (Sport, Musik, militärisch)}
\label{subsubsec:elite_training}
% - Hochmotivierte Populationen, kleinere Gruppen
% - Individuelle Betreuung möglich
% - Kosteneffizienz ggf. gegeben
% - Realistisch: 5-10 Jahre bis etablierte Praxis

% Subsubsection 3.3.1.2: Mainstream-Bildung (Schulen, Universitäten)
% \subsubsection{Mainstream-Bildung}
\label{subsubsec:mainstream_bildung}
% - Massive Skala (Millionen), Kostendruck
% - Heterogene Populationen (nicht alle profitieren)
% - Logistische Herausforderungen
% - Realistisch: 15-25 Jahre oder später für breite Integration

% Subsubsection 3.3.1.3: Home-Use für individuelle Lerner
% \subsubsection{Home-Use für Selbstoptimierung}
\label{subsubsec:homeuse_self}
% - Consumer-Geräte bereits verfügbar
% - Größtes Risiko: unsachgemäße Anwendung ohne Supervision
% - Dual-Use Problem: Medizinisch/Enhancement-Nutzung vermischt
