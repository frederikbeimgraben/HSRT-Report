%!TEX root = ../../main.tex
% ==============================================================================
% CHAPTER 2: WISSENSCHAFTLICHE EVIDENZ
% ==============================================================================
\chapter{Wissenschaftliche Evidenz für Enhancement durch Hirnstimulation}
\label{chap:evidenz}

% LITERATURVORSCHLÄGE FÜR KAPITEL:
% - Simonsmeier et al. (2018): "Electrical brain stimulation improves learning" [Meta-Analyse]
% - Reis et al. (2009): "Noninvasive cortical stimulation enhances motor skill acquisition" [motorisch, 1700+ Zit.]
% - Akkad et al. (2021): "Increasing human motor skill acquisition by driving theta-gamma" [innovative Studie]
% - Cavaleiro et al. (2020): "Memory and Cognition-Related Neuroplasticity" [Mechanismen]
% - Frontiers meta-analysis (2024): TMS effects on cognition
% - Vergallito et al. (2022): "Inter-Individual Variability in tDCS Effects" [Variabilität, 154 Zit.]

\section{Neurobiologische Wirkmechanismen}
\label{sec:wirkmechanismen}

% LITERATUR: Cavaleiro 2020, Woods 2016 technical guide, Niksche & Paulus reviews

% \subsection{Grundprinzipien der transkraniellen Hirnstimulation}
% \label{subsec:grundprinzipien}
% - Elektromagnetische Induktion
% - Modulation kortikaler Erregbarkeit
% - Unterschiede tDCS (konstant), tACS (oszillatorisch), TMS (magnetisch)
% - Relevante Hirnregionen (DLPFC, M1, parietal cortex)

% Subsubsection 2.1.1.1: tDCS Mechanismus
% \subsubsection{Transkranielle Gleichstromstimulation (tDCS)}
% \label{subsubsec:tdcs_mech}
% - Anodale vs. kathodale Stimulation
% - Membranpotenzial-Verschiebung
% - Stromfluss und elektrische Feldverteilung
% - Parameter: Intensität (1-2 mA), Dauer (20-30 min), Elektroden-Montage

% Subsubsection 2.1.1.2: tACS Mechanismus
% \subsubsection{Transkranielle Wechselstromstimulation (tACS)}
% \label{subsubsec:tacs_mech}
% - Frequenzabhängige Modulation
% - Entrainment neuronaler Oszillationen
% - Theta (4-8 Hz), Gamma (40 Hz), andere Bänder
% - Theta-Gamma-Kopplung für Gedächtnis/Lernen

% Subsubsection 2.1.1.3: TMS/rTMS Mechanismus
% \subsubsection{Transkranielle Magnetstimulation (TMS/rTMS)}
% \label{subsubsec:tms_mech}
% - Faradaysche Induktion
% - Aktionspotenziale in kortikalen Neuronen
% - High-frequency (>5 Hz) exzitatorisch, Low-frequency (<1 Hz) inhibitorisch
% - iTBS vs. cTBS Protokolle

% \subsection{Zelluläre und molekulare Plastizitätsmechanismen}
% \label{subsec:plastizität}

% LITERATUR: Cavaleiro 2020, Frontiers molecular neuroscience 2020, Esser 2006

% Subsubsection 2.1.2.1: Long-Term Potentiation (LTP)
% \subsubsection{Langzeitpotenzierung und -depression}
% \label{subsubsec:ltp_ltd}
% - LTP als synaptische Verstärkung
% - LTD als synaptische Abschwächung
% - iTBS induziert LTP-ähnliche Effekte
% - cTBS induziert LTD-ähnliche Effekte
% - Messung: MEP (Motor Evoked Potentials)

% Subsubsection 2.1.2.2: Neurotrophische Faktoren
% \subsubsection{Neurotrophische Faktoren und Genexpression}
% \label{subsubsec:neurotrophisch}
% - BDNF (Brain-Derived Neurotrophic Factor)
% - c-fos, zif268 (immediate early genes)
% - NMDA-Rezeptoren
% - Hochregulation durch Stimulation (besonders iTBS)

% Subsubsection 2.1.2.3: Transmitter-Modulation
% \subsubsection{Neurotransmitter-Modulation}
% \label{subsubsec:transmitter}
% - GABAerge und glutamaterge Systeme
% - Dopamin und Lernverhalten (Reward-based Learning)
% - Monoamine (Serotonin, Noradrenalin)
% - Inter-individuelle Variabilität durch Transmitter-Polymorphismen

% \subsection{Zeitliche Aspekte: Phasen des Lernens}
% \label{subsec:zeitliche_aspekte}

% LITERATUR: Reis 2009, Debarnot 2019, Zimerman 2013

% Subsubsection 2.1.3.1: Enkodierung und Online-Lernen
% \subsubsection{Enkodierung und Online-Lernen}
% \label{subsubsec:enkodierung}
% - Effekte während aktiver Aufgabe
% - Stimulation während Trainingsphase
% - Hemmt oder fördert Online-Lernen je nach Timing/Polarität
% - Studien: [Refs: Reis 2009 - motorisch, Stagg 2011]

% Subsubsection 2.1.3.2: Konsolidierung und Offline-Effekte
% \subsubsection{Konsolidierung und Offline-Effekte}
\label{subsubsec:konsolidierung}
% - Verstärkung zwischen Trainings-Sessions
% - Sleep-dependent consolidation
% - tDCS während Tiefschlaf: Slow Oscillations induzieren
% - Externe Effekte: 1h, 24h, 1 Woche post-stimulation
% - Studien: [Reis 2009, Marshall & Born 2007]

\section{Empirische Befunde nach kognitiver Domäne}
\label{sec:befunde_domänen}

% LITERATURVORSCHLÄGE: Simonsmeier 2018, Reis 2009, Akkad 2021, Lopez-Alonso 2025

% \subsection{Motorisches Lernen und Fertigkeitserwerb}
\label{subsec:motorisch}

% LITERATUR: Reis 2009 (1700 Zit.), Akkad 2021, Maceira-Elvira 2024, Lopez-Alonso 2025

% Subsubsection 2.2.1.1: Anodales tDCS über M1
% \subsubsection{Anodale tDCS über motorischem Kortex (M1)}
\label{subsubsec:motor_tdcs}
% - Effekte auf Konsolidierung (nicht Online-Lernen)
% - Protokoll: 2 mA, 20 min, posttraining
% - Effektgrößen: Cohen's d = 0.5-1.0
% - Persistenz: bis zu 24h
% - Alter-abhängig: auch ältere Probanden profitieren

% Subsubsection 2.2.1.2: Theta-Gamma-tACS über M1/sensomotorischer Kortex
% \subsubsection{Theta-Gamma-tACS für motorisches Lernen}
\label{subsubsec:motor_tacs}
% - Innovative Frequenzkopplung
% - Phase-Amplitude-Coupling (4-8 Hz Theta moduliert 40 Hz Gamma)
% - Replikationsstudie [Akkad 2021]
% - Effekte: Beschleunigung, Persistenz über 1h
% - Mechanismus: Theta-Gamma relevant für hippokampales Lernen

% Subsubsection 2.2.1.3: Komplexe motorische Skills
% \subsubsection{Komplexe motorische Skills (negative Befunde)}
\label{subsubsec:komplexe_motor}
% - Golf-Putting Studie [Lopez-Alonso 2025]
% - tDCS zeigte KEINE zusätzliche Verbesserung über Training
% - Implikation: Einfache vs. komplexe Skills unterschiedlich
% - Limitationen: Realwelt-Komplexität, Multi-Link Bewegungen

% \subsection{Akademische Fähigkeiten (Mathematik, Sprache)}
\label{subsec:akademisch}

% LITERATUR: Simonsmeier et al. 2018 (133 Zit., Meta-Analyse), Cohen Kadosh Arbeiten, Sarkar 2014

% Subsubsection 2.2.2.1: Timing der Stimulation
% \subsubsection{Timing: Stimulation während Lernphase vs. Test}
\label{subsubsec:timing_learning}
% - Simonsmeier Meta-Analyse [2018]: Stimulation während Lernen > während Test
% - Mechanismus: LTP-Induktion während aktiver Verarbeitung
% - Effektgrößen: SMD = 0.3-0.7 für akademische Fähigkeiten

% Subsubsection 2.2.2.2: Mathematische Kompetenzen
% \subsubsection{Mathematische Fähigkeiten}
\label{subsubsec:mathe}
% - tRNS (Random Noise) über "Mathezentrum" (DLPFC/IPS)
% - Schmerzfreie Alternative zu tDCS
% - Verbesserung Reaktionszeit + Genauigkeit
% - [Cohen Kadosh Arbeiten, erwähnt in Spektrum.de]

% Subsubsection 2.2.2.3: Sprachfähigkeiten
% \subsubsection{Sprachfähigkeiten}
\label{subsubsec:sprache}
% - Weniger Forschung als Mathematik
% - tDCS über Sprach-Arealen (Broca, Wernicke)
% - Stimulation während Vokabel-Lernen
% - Effektgrößen: moderate bis klein

% \subsection{Arbeitsgedächtnis und exekutive Funktionen}
\label{subsec:workingmem}

% LITERATUR: Hoy et al. 2016, Senkowski & Sobirey 2022, Imburgio & Orr 2018

% Subsubsection 2.2.3.1: tDCS vs. tACS Vergleich
% \subsubsection{Vergleich tDCS vs. tACS auf Arbeitsgedächtnis}
\label{subsubsec:tdcs_vs_tacs}
% - Systematische Review [Senkowski 2022]: 43 Studien, 1826 Teilnehmer
% - Single-Session tDCS: kaum signifikant
% - Multi-Session tDCS: moderate Effekte (SMD ~0.3)
% - tACS (v.a. Gamma): robustere Effekte (SMD ~0.4-0.6)
% - Frequenz-Abhängigkeit: Gamma (40 Hz) > andere

% Subsubsection 2.2.3.2: iTBS auf DLPFC
% \subsubsection{Intermittierende Theta-Burst-Stimulation (iTBS) auf DLPFC}
\label{subsubsec:itbs_dlpfc}
% - Hoy et al. (2016): iTBS > tDCS auf 3-back Working Memory Task
% - Theta & Gamma Synchronisation erhöht
% - Effektgrößen größer als tDCS
% - Schnellere Behandlung (10min statt 20min)

% Subsubsection 2.2.3.3: Exekutive Funktionen
% \subsubsection{Exekutive Funktionen (Inhibition, Flexibilität)}
\label{subsubsec:exek_funkt}
% - tDCS Effekte: limitiert auf Refresh (WM)
% - Inhibition: kleine bis keine Effekte [Imburgio & Orr 2018]
% - Kognitive Flexibilität: wenig untersucht
% - Kontext-abhängigkeit der Effekte

% \subsection{Praktische Anwendungen in der Realwelt}
\label{subsec:realwelt_apps}

% LITERATUR: Clark et al. 2012, Lopez-Alonso 2025, Violante 2024

% Subsubsection 2.2.4.1: Threat-Detection-Training (Militär)
% \subsubsection{Threat-Detection-Training}
\label{subsubsec:threat_detection}
% - Clark et al. (2012): Virtual Reality + tDCS
% - 96 Probanden, Erkennung versteckter Objekte
% - Signifikante Verbesserung + Trainingstransfer
% - DARPA finanzierte Forschung
% - Praktische Implikation: funktioniert in komplexer Umgebung

% Subsubsection 2.2.4.2: Negative Befunde und Realwelt-Limitationen
% \subsubsection{Negative Befunde und Realwelt-Limitationen}
\label{subsubsec:negative_realwelt}
% - Lopez-Alonso (2025): Golf-Putting keine zusätzliche Verbesserung
% - Violante et al. (2024): "modest evidence" für Neuro-Enhancement
% - Komplexe Skills brauchen Integration verschiedener Systeme
% - Transfer-Problem: Lab ≠ Realwelt

\section{Kritische Limitationen der Evidenz}
\label{sec:limitationen}

% LITERATURVORSCHLÄGE: Vergallito 2022, Chew 2015, Maceira-Elvira 2024, Violante 2024

% \subsection{Inter-individuelle Variabilität in Stimulationsresponse}
\label{subsec:variabilität}

% LITERATUR: Vergallito et al. 2022 (154 Zit.), Chew et al. 2015 (317 Zit.), Filmer et al. 2014

% Subsubsection 2.3.1.1: Anatomische Unterschiede
% \subsubsection{Anatomische Unterschiede}
\label{subsubsec:anat_diff}
% - Kortikale Gyrierung (Gyri/Sulci Patterns)
% - Kortexdicke in spezifischen Regionen
% - Faserverbindungen zwischen Regionen
% - Vorhersage: Dicke in mittlerem Stirnfurchen prädiziert tDCS-Response [Filmer]
% - Modellierung: FEM (Finite-Elemente) kann Feldverteilung individualisieren

% Subsubsection 2.3.1.2: Genetische und biochemische Faktoren
% \subsubsection{Genetische und biochemische Faktoren}
\label{subsubsec:genetik}
% - BDNF val66met Polymorphismus
% - Transmitter-Polymorphismen (COMT, DAT)
% - Baseline-GABA/Glutamat Ratio
% - Limitierte Forschung, aber vielversprechend für Personalisierung

% Subsubsection 2.3.1.3: Funktionelle Baseline-Unterschiede
% \subsubsection{Funktionelle Baseline-Unterschiede}
\label{subsubsec:baseline_func}
% - Resting-State Konnektivität
% - Theta-Power im Frontallappen (prädiziert rTMS-Response bei Depression)
% - Baseline-Leistung auf kognitiver Aufgabe

% \subsection{Baseline-Leistung und Lerner-Typ}
\label{subsec:baseline_leistung}

% LITERATUR: Maceira-Elvira 2024, Sarkar 2014

% Subsubsection 2.3.2.1: Wer profitiert am meisten?
% \subsubsection{Baseline-abhängige Responsiveness}
\label{subsubsec:baseline_response}
% - Maceira-Elvira (2024): "Native learning ability, not age, determines response"
% - Personen mit niedrigen Ausgangsfähigkeiten profitieren MEHR
% - Hochleister: KEINE zusätzliche Verbesserung durch tDCS
% - Implikation: Enhancement-Effekt abhängig vom Starting Point

% Subsubsection 2.3.2.2: Math-Anxiety Beispiel
% \subsubsection{Kontext-abhängige Effekte: Math-Anxiety}
\label{subsubsec:math_anxiety}
% - Sarkar et al. (2014): tDCS hilft bei Mathe-Angst
%   - Verbesserung Reaktionszeit + reduziertes Cortisol
% - Aber: Bei Nicht-Angst-Personen verschlechtert sich Leistung!
% - Implikation: Effekte nicht universal, stark kontext-abhängig

% \subsection{Nur 39-45\% sind Responder}
\label{subsec:responder_rate}

% LITERATUR: Vergallito 2022, Chew 2015, Filmer 2014

% Subsubsection 2.3.3.1: Empirische Evidenz
% \subsubsection{Responder vs. Non-Responder Phänomenologie}
\label{subsubsec:responder_phenom}
% - Vergallito (2022): Nur 39-45% zeigen erwartete Effekte
% - Cluster-Analysen zeigen bimodale Verteilungen
% - Chew (2015): Intra-individual Variability ICC = -0.50 (negligible!)
% - Problem: Nicht zuverlässig vorhersagbar mit Current Protocols

% Subsubsection 2.3.3.2: Implikationen für praktische Anwendung
% \subsubsection{Implikationen für praktische Anwendung}
\label{subsubsec:responder_impl}
% - Screening notwendig zur Identification von Respondern
% - Adaptives Protokoll-Adjustments
% - Oder: Personalisierung VOR Behandlung (individuelles FEM-Modell, fMRI)

% \subsection{Transferdefizite und Aufgabenspezifität}
\label{subsec:transfer}

% LITERATUR: Violante 2024, Lopez-Alonso 2025

% Subsubsection 2.3.4.1: Begrenzte Generalisierbarkeit
% \subsubsection{Begrenzte Generalisierbarkeit der Effekte}
\label{subsubsec:generalisierbar}
% - Effekte meist aufgabenspezifisch
% - Transfer zu anderen Aufgaben limitiert
% - Transfer zu anderen Domänen: unklar
% - Implikation: Nicht "allgemeine Intelligenz-Verbesserung"

% Subsubsection 2.3.4.2: Realwelt-Transfer
% \subsubsection{Transfer von Lab zu Realwelt}
\label{subsubsec:lab_realwelt}
% - Threat-Detection erfolgreich [Clark 2012]
% - Golf-Putting KEIN Transfer [Lopez-Alonso 2025]
% - Komplexität des Realwelt-Skills kritisch
% - Context-Dependence der Stimulation-Effekte

% \subsection{Fehlende Langzeitstudien und Persistenz}
\label{subsec:langzeit}

% LITERATUR: Brühl 2019, Woodham 2024 (nur 10 Wochen), Violante 2024

% Subsubsection 2.3.5.1: Kurzfristige vs. Langfristige Effekte
% \subsubsection{Persistenz von Effekten}
\label{subsubsec:persistenz}
% - Kurzfristig: Gut dokumentiert (bis 24h post-stimulation)
% - Mittelfristig (1-4 Wochen): Noch nicht vollständig untersucht
% - Langfristig (Monate/Jahre): Kaum Daten
% - Frage: Was passiert nach Beendigung der Stimulation?

% Subsubsection 2.3.5.2: Lifetime Safety bei Gesunden
% \subsubsection{Lifetime Safety bei wiederholter Anwendung bei Gesunden}
\label{subsubsec:lifetime_safety}
% - Kurzfristige Nebenwirkungen: gut dokumentiert, mild
% - Langfristige Effekte unbekannt [Brühl 2019]
% - Besondere Bedenken: Hirnentwicklung (<25 Jahre)
% - Frage: Könnte wiederholte Stimulation Neuroplastizität negativ beeinflussen?
% - Forschungsbedarf: Prospektive Langzeitstudien
