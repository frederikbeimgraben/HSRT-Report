% !TEX root = ../../Main.tex

% Meeting date
\newcommand{\waterMarkText}{}

% Current date and time
\createdon{XX.XX.20XX}

% Title
\title{Vorgaben zur Formatierung der Seminararbeit}

% Data fields for the title page
\AddTitlePageDataLine{Thema}{
	Thema-XXX: \newline
	Vorgaben zur Formatierung der Seminararbeit
}
\AddTitlePageDataSpace{5pt}
\AddTitlePageDataLine{Vorgelegt von}{
	Hans Maria Muster \newline
	X. Fachsemester \newline
	\href{mailto:hans-maria.muster@student.hs-reutlingen.de}{hans-maria.muster@student.hs-reutlingen.de}
}
\AddTitlePageDataSpace{5pt}
\AddTitlePageDataLine{Vorgelegt am}{XX.XX.20XX}
\AddTitlePageDataSpace{5pt}
\AddTitlePageDataLine{Studiengang}{Medizinisch Technische Informatik B.Sc.}
\AddTitlePageDataLine{Modul}{
	METI6.3 \newline
	Seminar Ausgewählter Themen der Medizinisch-Technischen Informatik
}
\AddTitlePageDataLine{Dozent:in}{Prof. Dr. Sven Steddin}
\AddTitlePageDataLine{Semester}{Wintersemester 20XX/20XX}
\AddTitlePageDataSpace{5pt}
\AddTitlePageDataLine{Wortanzahl}{\quickwordcount{Content/01_content}} % !!! Only 01_content.tex !!!


% Abstract
\newcommand{\titlepageabstract}{%
	Das Abstract beschreibt in wenigen Sätzen die Zielsetzung und das Ergebnis der Ausarbeitung. Das Abstract muss sich vollständig auf der Titelseite befinden. Die Zeichensatzformatierung wird in einem eigenen Absatz beschrieben  Das Abstract soll es den Lesern:innen ermöglichen, innerhalb von wenigen Augenblicken zu erfassen, welcher Inhalt hinter der Überschrift steckt und ob das Thema, aus Sicht der Leser:innen, zur weiteren Bearbeitung lohnt. Das Abstract ist keine verbale Beschreibung des Inhaltsverzeichnisses, sondern gibt kurz und knapp z.B. die Zielsetzung (z.B. Hypothese), die eingesetzten Methoden und die erzielten Ergebnisse / Erkenntnisse bekannt. Weitere Hinweise finden Sie außerdem im Vorlesungsskript.
}

% Keywords
\newcommand{\titlepagekeywords}{%
	Seminararbeit, wissenschaftliche Ausarbeitung, Bachelor-Thesis, Studium, Plagiat
}

% List of equations
\renewcommand{\listequationsname}{Formeln und Gleichungen}
\renewcommand{\equationname}{Formel}

% Disable indentation
\setlength{\parindent}{0pt}
