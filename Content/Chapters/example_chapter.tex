% !TEX root = ../../Main.tex
% ==============================================================================
% Example Chapter Template
% ==============================================================================
% Description: This file serves as a template for creating new chapters.
%              It demonstrates the proper structure, formatting, and various
%              LaTeX elements that can be used in academic documents.
% Usage:       Copy this file and rename it for your own chapters.
%              Include it in the main document using \input{Content/Chapters/your_chapter}
% Author:      [Your Name]
% Date:        [Date]
% ==============================================================================

% ------------------------------------------------------------------------------
% Chapter Declaration
% ------------------------------------------------------------------------------
% Start a new chapter with a descriptive title
% The \label command allows for cross-referencing
% ------------------------------------------------------------------------------
\chapter{Example Chapter Title}
\label{chap:example_chapter}

% ------------------------------------------------------------------------------
% Chapter Introduction
% ------------------------------------------------------------------------------
% Begin with an introductory paragraph that outlines the chapter's content
% This helps readers understand what to expect
% ------------------------------------------------------------------------------

This chapter demonstrates the various features and formatting options available
in the HSRTReport template. It serves as a comprehensive guide for creating
well-structured academic content. The following sections will cover different
aspects of document formatting, from basic text elements to complex mathematical
expressions and graphical content.

% ==============================================================================
% Section: Basic Text Formatting
% ==============================================================================
\section{Basic Text Formatting}
\label{sec:basic_formatting}

% ------------------------------------------------------------------------------
% Subsection: Text Styles
% ------------------------------------------------------------------------------
\subsection*{Text Styles}
\label{subsec:text_styles}

The template supports various text formatting options:

\begin{itemize}
	\item \textbf{Bold text} for emphasis on important terms
	\item \textit{Italic text} for foreign words or emphasis
	\item \texttt{Monospace text} for code, file names, or commands
	\item \underline{Underlined text} for special highlighting
	\item \textsc{Small Caps} for acronyms or special names
	\item Combined styles: \textbf{\textit{bold and italic}}
\end{itemize}

% ------------------------------------------------------------------------------
% Subsection: Quotations
% ------------------------------------------------------------------------------
\subsection*{Quotations and Citations}
\label{subsec:quotations}

For short inline quotes, use regular quotation marks: ``This is a short quote.''

For longer quotations, use the quote environment:

\begin{quote}
	``This is a longer quotation that demonstrates how block quotes are formatted
	in the template. Block quotes are typically indented from both margins and
	may have different spacing. They are ideal for quotations that exceed
	three lines of text.'' \cite{example2023}
\end{quote}

% Direct citation with page number
According to \textcite[p.~42]{examplebook2022}, the methodology is sound.

% Multiple citations
Several studies have confirmed these findings \parencite{example2023, conference2023}.

% ==============================================================================
% Section: Lists and Enumerations
% ==============================================================================
\section{Lists and Enumerations}
\label{sec:lists}

% ------------------------------------------------------------------------------
% Subsection: Unordered Lists
% ------------------------------------------------------------------------------
\subsection*{Unordered Lists}
\label{subsec:unordered_lists}

Simple bullet points:

\begin{itemize}
	\item First main point
	\item Second main point
	\item Third main point with sub-items:
	      \begin{itemize}
		      \item First sub-item
		      \item Second sub-item
		            \begin{itemize}
			            \item Nested sub-sub-item
			            \item Another nested item
		            \end{itemize}
	      \end{itemize}
	\item Final main point
\end{itemize}

% ------------------------------------------------------------------------------
% Subsection: Ordered Lists
% ------------------------------------------------------------------------------
\subsection*{Ordered Lists}
\label{subsec:ordered_lists}

Numbered lists for sequential items:

\begin{enumerate}
	\item First step in the process
	\item Second step with sub-steps:
	      \begin{enumerate}
		      \item Sub-step 2.1
		      \item Sub-step 2.2
	      \end{enumerate}
	\item Third step
	\item Final step
\end{enumerate}

% ------------------------------------------------------------------------------
% Subsection: Description Lists
% ------------------------------------------------------------------------------
\subsection*{Description Lists}
\label{subsec:description_lists}

For term definitions:

\begin{description}
	\item[Term One] Definition or description of the first term. This can be
		multiple lines long and will be properly formatted.
	\item[Term Two] Definition of the second term with emphasis on clarity.
	\item[Complex Term] A more detailed explanation that might include
		technical details, examples, or references to other concepts.
\end{description}

% ==============================================================================
% Section: Mathematical Content
% ==============================================================================
\section{Mathematical Expressions}
\label{sec:mathematics}

% ------------------------------------------------------------------------------
% Subsection: Inline Mathematics
% ------------------------------------------------------------------------------
\subsection*{Inline Mathematics}
\label{subsec:inline_math}

Mathematics can be included inline, such as the quadratic formula
$x = \frac{-b \pm \sqrt{b^2 - 4ac}}{2a}$, or the famous Euler's identity
$e^{i\pi} + 1 = 0$.

Common mathematical symbols include:
\begin{itemize}
	\item Greek letters: $\alpha$, $\beta$, $\gamma$, $\Delta$, $\Omega$
	\item Operations: $\sum_{i=1}^{n}$, $\prod_{j=1}^{m}$, $\int_{a}^{b}$
	\item Relations: $\leq$, $\geq$, $\approx$, $\equiv$, $\propto$
	\item Sets: $\mathbb{R}$, $\mathbb{N}$, $\mathbb{C}$, $\emptyset$
\end{itemize}

% ------------------------------------------------------------------------------
% Subsection: Display Mathematics
% ------------------------------------------------------------------------------
\subsection*{Display Mathematics}
\label{subsec:display_math}

For important equations, use the equation environment:

\begin{equation}
	\nabla \times \vec{B} - \frac{1}{c^2}\frac{\partial \vec{E}}{\partial t}
	= \mu_0 \vec{J}
	\label{eq:maxwell}
\end{equation}
\myequations{Maxwell's Equations}

Equation \eqref{eq:maxwell} represents one of Maxwell's equations.

Multi-line equations with alignment:

\begin{align}
	f(x) & = x^2 + 2x + 1 \label{eq:function} \\
	     & = (x + 1)^2 \label{eq:factored}
\end{align}

Matrix example:

\begin{equation}
	\mathbf{A} = \begin{bmatrix}
		a_{11} & a_{12} & \cdots & a_{1n} \\
		a_{21} & a_{22} & \cdots & a_{2n} \\
		\vdots & \vdots & \ddots & \vdots \\
		a_{m1} & a_{m2} & \cdots & a_{mn}
	\end{bmatrix}
	\label{eq:matrix}
\end{equation}
\myequations{Matrix}

% ==============================================================================
% Section: Figures and Graphics
% ==============================================================================
\section{Figures and Graphics}
\label{sec:figures}

% ------------------------------------------------------------------------------
% Subsection: Single Figures
% ------------------------------------------------------------------------------
\subsection*{Single Figures}
\label{subsec:single_figures}

\begin{figure}[htbp]
	\centering
	% Replace the rule with actual image:
	% \includegraphics[width=0.6\textwidth]{Content/Images/your_image.png}
	\rule{0.6\textwidth}{0.4\textwidth} % Placeholder
	\caption[Short caption for list of figures]{Detailed caption explaining the
		figure content. This caption can be quite long and should provide
		sufficient information for understanding the figure without reading
		the main text.}
	\label{fig:single_example}
\end{figure}

As shown in Figure~\ref{fig:single_example}, the visualization clearly
demonstrates the concept. Always reference figures in the text.

% ------------------------------------------------------------------------------
% Subsection: Multiple Figures
% ------------------------------------------------------------------------------
\subsection*{Side-by-Side Figures}
\label{subsec:multiple_figures}

\begin{figure}[htbp]
	\centering
	\begin{subfigure}[b]{0.45\textwidth}
		\centering
		% \includegraphics[width=\textwidth]{Content/Images/image1.png}
		\rule{\textwidth}{0.6\textwidth} % Placeholder
		\caption{First subfigure}
		\label{fig:sub1}
	\end{subfigure}
	\hfill
	\begin{subfigure}[b]{0.45\textwidth}
		\centering
		% \includegraphics[width=\textwidth]{Content/Images/image2.png}
		\rule{\textwidth}{0.6\textwidth} % Placeholder
		\caption{Second subfigure}
		\label{fig:sub2}
	\end{subfigure}
	\caption{Comparison of two related concepts: (a) shows the first concept,
		while (b) illustrates the second concept.}
	\label{fig:comparison}
\end{figure}

% ==============================================================================
% Section: Tables
% ==============================================================================
\section{Tables}
\label{sec:tables}

% ------------------------------------------------------------------------------
% Subsection: Simple Tables
% ------------------------------------------------------------------------------
\subsection*{Simple Tables}
\label{subsec:simple_tables}

\begin{table}[htbp]
	\centering
	\caption{Comparison of different methods}
	\label{tab:comparison}
	\begin{tabular}{lccc}
		\toprule
		\textbf{Method} & \textbf{Accuracy (\%)} & \textbf{Speed (ms)} & \textbf{Memory (MB)} \\
		\midrule
		Method A        & 95.2                   & 12.3                & 256                  \\
		Method B        & 97.8                   & 45.6                & 512                  \\
		Method C        & 93.1                   & 8.9                 & 128                  \\
		Method D        & 99.1                   & 78.2                & 1024                 \\
		\bottomrule
	\end{tabular}
\end{table}

Table~\ref{tab:comparison} presents a performance comparison of different methods.

% ------------------------------------------------------------------------------
% Subsection: Complex Tables
% ------------------------------------------------------------------------------
\subsection*{Complex Tables with Multicolumn}
\label{subsec:complex_tables}

\begin{table}[htbp]
	\centering
	\caption{Experimental results across different conditions}
	\label{tab:experimental}
	\begin{tabular}{l|cc|cc}
		\toprule
		\multirow{2}{*}{\textbf{Sample}}          &
		\multicolumn{2}{c|}{\textbf{Condition A}} &
		\multicolumn{2}{c}{\textbf{Condition B}}                                                              \\
		                                          & \textbf{Mean} & \textbf{SD} & \textbf{Mean} & \textbf{SD} \\
		\midrule
		Sample 1                                  & 12.3          & 1.2         & 15.6          & 2.1         \\
		Sample 2                                  & 13.5          & 1.5         & 16.8          & 1.9         \\
		Sample 3                                  & 11.9          & 1.1         & 14.2          & 2.3         \\
		\midrule
		\textbf{Average}                          & 12.6          & 1.3         & 15.5          & 2.1         \\
		\bottomrule
	\end{tabular}
\end{table}

% ==============================================================================
% Section: Code Listings
% ==============================================================================
\section{Source Code}
\label{sec:source_code}

% ------------------------------------------------------------------------------
% Subsection: Code Examples
% ------------------------------------------------------------------------------
\subsection*{Programming Examples}
\label{subsec:code_examples}

\begin{lstlisting}[language=Python, caption={Python function example}, label=lst:python_example]
def calculate_statistics(data):
    """
    Calculate mean and standard deviation of a dataset.

    Args:
        data (list): List of numerical values

    Returns:
        tuple: Mean and standard deviation
    """
    import numpy as np

    mean = np.mean(data)
    std = np.std(data)

    return mean, std

# Example usage
dataset = [1, 2, 3, 4, 5]
mean, std = calculate_statistics(dataset)
print(f"Mean: {mean:.2f}, Std: {std:.2f}")
\end{lstlisting}

Listing~\ref{lst:python_example} demonstrates a simple statistical calculation
function in Python.

% ==============================================================================
% Section: Cross-References and Hyperlinks
% ==============================================================================
\section{Cross-References}
\label{sec:cross_references}

% ------------------------------------------------------------------------------
% Subsection: Internal References
% ------------------------------------------------------------------------------
\subsection*{Internal References}
\label{subsec:internal_refs}

This template supports various types of cross-references:

\begin{itemize}
	\item Chapters: See Chapter~\ref{chap:example_chapter}
	\item Sections: As discussed in Section~\ref{sec:mathematics}
	\item Subsections: Details in Subsection~\ref{subsec:text_styles}
	\item Figures: Shown in Figure~\ref{fig:single_example}
	\item Tables: Data in Table~\ref{tab:comparison}
	\item Equations: According to Equation~\eqref{eq:maxwell}
	\item Pages: On page~\pageref{sec:lists}
\end{itemize}

% ------------------------------------------------------------------------------
% Subsection: External Links
% ------------------------------------------------------------------------------
\subsection*{External Links}
\label{subsec:external_links}

Links to external resources:
\begin{itemize}
	\item URL: \url{https://www.example.com}
	\item Email: \href{mailto:email@example.com}{email@example.com}
	\item Linked text: \href{https://www.latex-project.org}{LaTeX Project}
\end{itemize}

% ==============================================================================
% Section: Glossary and Acronyms
% ==============================================================================
\section{Using Glossary and Acronyms}
\label{sec:glossary_usage}

% Demonstrate glossary usage
The \gls{Textkoerper} contains the main content. When using acronyms like
\acrfull{a:MPG}, they are automatically formatted. Subsequent uses show only
the short form: \acrshort{a:MPG}.

% Add more glossary entries as needed
Technical terms should be defined in the glossary file and referenced using
the appropriate commands throughout the document.

% ==============================================================================
% Section: Advanced Features
% ==============================================================================
\section{Advanced Features}
\label{sec:advanced}

% ------------------------------------------------------------------------------
% Subsection: Footnotes and Margin Notes
% ------------------------------------------------------------------------------
\subsection*{Footnotes}
\label{subsec:footnotes}

Footnotes provide additional information\footnote{This is an example footnote
	that contains supplementary information not essential to the main text.} without
interrupting the flow of the main text.

% ------------------------------------------------------------------------------
% Subsection: Theorems and Proofs
% ------------------------------------------------------------------------------
\subsection*{Theorems and Definitions}
\label{subsec:theorems}

\begin{definition}[Important Concept]
	A formal definition of an important concept that requires precise
	mathematical or technical language.
\end{definition}

\begin{theorem}[Fundamental Theorem]
	\label{thm:fundamental}
	Statement of an important theorem that will be proven or referenced.
\end{theorem}

\begin{proof}
	The proof follows from the definition and basic principles...
	\[
		\text{QED}
	\]
\end{proof}

% ==============================================================================
% Chapter Summary
% ==============================================================================
\section{Chapter Summary}
\label{sec:chapter_summary}

This chapter has demonstrated the key features of the HSRTReport template,
including:

\begin{itemize}
	\item Text formatting and typography options
	\item Mathematical expressions and equations
	\item Figures, tables, and code listings
	\item Cross-referencing and citations
	\item Advanced document features
\end{itemize}

These elements form the foundation for creating professional academic documents
using this template. Each feature can be customized further to meet specific
requirements of your document.

% ==============================================================================
% End of Example Chapter
% ==============================================================================
