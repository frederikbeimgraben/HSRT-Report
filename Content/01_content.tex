% !TEX root = ../Main.tex

% === TexCount-Definitions ===
% (Ignore Headings)
%TC:macro \chapter [ignore]
%TC:macro \subsection* [ignore]
%TC:macro \subsection [ignore]
%TC:macro \subsubsection [ignore]
%TC:macro \includesvg [ignore]
% (Ignore Figures and Tables)
%TC:envir figure [ignore] ignore
%TC:envir table [ignore] ignore

% === Content ===
\newpage
\chapter{Einleitung}
Das hier vorliegende Dokument soll als Vorlage für die Gestaltung der im Seminarkurs Ausgewählte Themen der Medizinisch-Technischen Informatik erstellten schriftlichen Ausarbeitungen dienen.
Es ist zu beachten, dass das Formatieren des Textes als einer der letzten Schritte der Ausarbeitung durchgeführt wird, da dieser Schritt erfahrungsgemäß viel Zeit in Anspruch nimmt und daher nur einmalig ausgeführt werden sollte.
Der vorliegende Entwurf wurde mit Microsoft Office 2016 erstellt.
Alle in diesem Dokument enthaltenen Hinweise zur Gestaltung des Dokumentes dürfen im daraus abgeleiteten Dokument nicht mehr enthalten sein und müssen somit gelöscht werden.

\chapter{Gliederung des Textes}
Nach jedem der nachfolgend genannten Abschnitte muss mit einer neuen Seite begonnen werden. Die Ausarbeitung muss die nachfolgend gegebene Gliederung aufweisen (Hauptüberschriften).
Die als Liste angegeben Stichpunkte beschreiben, welche Inhalte in den Abschnitten behandelt werden sollten. Diese Stichpunkte sind, mit Ausnahme des Abschnitts „Verzeichnisse“, nicht zwingend als Unterüberschriften vorgegeben, können aber in gleicher oder ähnlicher Form verwendet werden, sofern dies sinnvoll erscheint. Achten Sie hierbei insbesondere darauf, dass ein mit einer Überschrift versehener Textblock nicht nur aus einem oder wenigen Sätzen bestehen darf.

\subsection*{Titelseite mit Abstract}
\subsection*{Verzeichnisse}

\begin{itemize}
	\item Inhaltsverzeichnis
	\item Abbildungsverzeichnis
	\item Tabellenverzeichnis
	\item Formelverzeichnis
	\item Abkürzungsverzeichnis
	\item Glossar
\end{itemize}

\subsection*{Einleitung}

\begin{itemize}
	\item Beschreibung des Problems und der Forschungsaufgabe
	\item Verdeutlichung der Relevanz für Wissenschaft und Gesellschaft
	\item Hypothese
	\item Definition der Leitfragen
	\item Stand der Wissenschaft / Technik
	\item Vorgehensweise zur Verifikation der Hypothese
\end{itemize}

\subsection*{Methoden}

\begin{itemize}
	\item Beschreibung der Umsetzung der Vorgehensweise zur Erzielung der gesuchten Ergebnisse (z.B. Aufbau der Messtechnik und Ablauf der Experimente, Beschreibung der Arbeitsinstrumente und Werkzeuge, Beschreibung der Lösungsprozesse oder Vorgehensweise bei der Literaturrecherche / Selektionskriterien, …)
	\item Methoden müssen so beschrieben sein, dass andere Personen das Verfahren nachvollziehen/reproduzieren können.
\end{itemize}

\subsection*{Ergebnisse}

\begin{itemize}
	\item Darstellung der über die Untersuchungsmethoden erzielten Ergebnisse (objektive Darstellung)
	\item Stellungnahme zur Verifikation der Hypothese durch Beantwortung der Leitfragen
\end{itemize}

\subsection*{Diskussion}

\begin{itemize}
	\item Interpretation der Ergebnisse (was lässt sich aus den Daten folgern à subjektive Beurteilung / persönliche Meinung)
	\item Vergleich der Ergebnisse mit den bisher bekannten Daten (Bewertung der Ergebnisse aus der Literatur: gibt es Übereinstimmung oder Widersprüche; wie lassen sich mögliche Widersprüche erklären?)
\end{itemize}

\subsection*{Zusammenfassung}

\begin{itemize}
	\item Kurze Beschreibung der Fragestellung und der Ergebnisse
	\item Ausblick und Empfehlungen
\end{itemize}

\subsection*{Literaturverzeichnis}

\subsection*{Danksagungen}

\begin{itemize}
	\item Benennung der Sponsoren
	\item Nennung der Hilfspersonen und deren Aufgabengebiet
\end{itemize}

\subsection*{Eidesstattliche Erklärung}

\begin{itemize}
	\item Der hier im Dokument gegebene Text muss übernommen werden.
	\item Das Dokument muss unterschrieben werden (z.B. durch die Verwendung der pdf-Unterschriftsfunktion im Acrobat Reader).
\end{itemize}

\subsection*{Anhang}

\begin{itemize}
	\item Der hier im Dokument gegebene Text muss übernommen werden.
\end{itemize}

\newpage
\chapter{Seitenformatierung}

\iffalse
	3	Seitenformatierung
	3.1	Seitenformat
		DinA4, hochkant
	3.2	Seitenränder
		Linker Rand:   4 cm
		Rechter Rand:   3 cm
		Oberer Rand:   3 cm
		Unterer Rand:   2 cm
	Jeder neue Absatz hat zum vorhergehenden Absatz einen Abstand von 6 Punkten. Für Listen (siehe letzter Absatz) gibt es die Formatvorlage „Listenabsatz“, bei der kein Abstand zwischen den Absätzen enthalten ist.
	Die Ausrichtung des Textes und aller Überschriften erfolgt linksbündig mit Flattersatz. Nach Möglichkeit muss die Silbentrennung eingesetzt werden, damit der Flattersatz den Gesamteindruck der Druckseite nicht zu sehr beeinträchtigt. Es wird empfohlen, zur Silbentrennung sogenannte bedingte Trennstriche einzusetzen.
	3.3	Titel und Überschriften
	Titel und Überschriften müssen kurz und aussagekräftig sein. Überschriften dürfen sich nicht wiederholen, sondern müssen eindeutig voneinander zu unterscheiden sein.
	Die Gliederung des Textes sollte maximal über 2 Ebenen erfolgen. Es ist darauf zu achten, dass nicht jeder Absatz eine eigene Kapitelüberschrift erhält (siehe oben).
	3.4	Textabsätze
	Ein Absatz darf nie mit einer einzelnen Zeile auf einer neuen Seite enden oder mit einer einzelnen Zeile am Seitenende beginnen. Dies gilt in gleicher Weise für Überschriften.
	Die Länge eines Absatzes ist auf maximal 400 Zeichen zu beschränken.
	Absätze werden nicht willkürlich gesetzt. Sie sollen dazu dienen das Dokument nicht nur formal, sondern auch inhaltlich zu gliedern und somit das Lesen und Verstehen des Textes erleichtern.
	Die Silbentrennung ist so einzusetzen, dass jede Zeile möglichst bis zum Zeilenrand beschrieben ist. Beim Einsatz von Blocksatz ist darauf zu achten, dass die Lücken zwischen den Worten nicht zu groß werden.
	Zwischen den einzelnen Absätzen ist ein Abstand von 6 Punkten einzuhalten.
	3.5	Kopfzeile
	Der Aufbau der Kopfzeile erfolgt, mit Ausnahme der Titelseite, gemäß dem in diesem Dokument enthaltenen Beispiel und beinhaltet auch die Kapitelnummer und –überschrift der ersten Ebene.
	3.6	Fußzeile
	Der Aufbau der Fußzeile erfolgt, mit Ausnahme der Titelseite, gemäß dem in diesem Dokument enthaltenen Beispiel.
	3.7	Seitennummerierung
	Die Seitennummerierung erfolgt entsprechend der Darstellung in diesem Dokument. Die Zählung beginnt bei 0 für das Titelblatt, so dass die erste Verzeichnisseite die Nummer 1 aufweist.
	3.8	Fußnoten
	Fußnoten werden im Text durch eine hochgestellte Ziffer grundsätzlich nach einem Satzzeichen (Komma, Strichpunkt, Punkt, Fragezeichen, Ausrufungszeichen, Gedankenstrich) referenziert. Die Fußnotennummerierung erfolgt fortlaufend über die gesamte Arbeit.
	Erläuterungen zu den Fußnoten erfolgen oberhalb der Fußzeile und reduzieren die Zeilenzahl des Textkörpers. Der Beginn einer Fußnote wird durch einen linksbündigen horizontalen Strich, dessen Länge ca. 30% Breite des Textblockes beträgt, vom Textkörper abgetrennt.
	Die Fußnotentexte sollten nicht länger als 4 Zeilen sein. Ebenso sollte vermieden werden, den Umbruch der Fußnoten auf die nächste Seite zu erzwingen.
	3.9	Formeln
	Formeln werden vom Rand einheitlich um 1 cm eingerückt und vom vorhergehenden und nachfolgenden Absatz um zwei Zeilen abgesetzt. Jede Formel ist über eine fortlaufend aufsteigend zu vergebende Nummer zu kennzeichnen. Bei Formeln, die durch Umformung auseinander hervorgehen, können die Zwischenschritte durch eine einheitliche Nummer, ergänzt um einen Buchstaben, referenziert werden.
	Beispiel:
	x=(-b±√(b^2-4ac))/( 2a)
	Formel 1 : Polstellenberechnung
	Um die Platzierung der automatischen Referenzierung der Formel auf der rechten Seite der Formel platzieren zu können, kann, wie oben gezeigt, eine Tabelle ohne sichtbare Ränder eingesetzt werden
	3.10	Abbildungen
	Abbildungen können an beliebiger Stelle im Text eingebaut werden. Jede Abbildung erhält eine Nummer, die sich aus der Abschnittsnummer der Ebene 1 und einer fortlaufenden, in jedem Abschnitt bei Eins beginnenden Nummer ergibt. Der Nummer folgt eine kurze Beschreibung zum Bild, die unterhalb des Bildes platziert wird. Die Nummer kann in Verbindung mit der Abkürzung Abb. oder dem Wort Abbildung zur Referenzierung einer Abbildung im Text eingesetzt werden. Hierzu werden im Text runde Klammern verwendet, z.B. (Abbildung 1) oder (Abb. 23).
	Alle Abbildungsnummern und – überschriften werden im Abbildungsverzeichnis referenziert. In Microsoft Word 2013 kann die Erstellung eines Abbildungsverzeichnisses automatisiert werden. Hierzu wird die Formatvorlage Beschriftung zur Verfügung gestellt.



	Werden Abbildungen aus fremden Quellen übernommen, so müssen die Copyrightvorgaben berücksichtigt werden, d.h. die Quelle der Abbildung muss angegeben und im Literaturverzeichnis referenziert werden. Für alle Abbildungen ohne Referenz beansprucht der Autor / die Autorin die eigene Urheberschaft. Dementsprechend dürfen selbst erstellte Abbildungen nicht mit Angaben wie „eigene Abbildung“ o.ä. versehen werden.
	3.11	Tabellen
	Tabellen können an beliebiger Stelle im Text eingebaut werden. Die erste Zeile einer Tabelle enthält die Spaltenbeschriftung, die erste Spalte ggf. die Zeilenbeschriftung.
	Es ist darauf zu achten, dass innerhalb einer Tabellenzeile kein Seitenumbruch erfolgt. Ist es erforderlich einen Seitenumbruch zwischen den Zeilen durchzuführen, so muss auf der nächsten Seite die Beschriftung der Spalten wiederholt werden.

	Formatbezeichner	Schrift	Größe	kursiv
	Überschrift 1	Franklin Gothic Book	16	nein
	Eigennamen	Times New Roman	12	ja
	Tabelle 3-1 : Beispiele für Formatvorlagen
	Auch die Tabellen werden in gleicher Form wie die Abbildungen nummeriert und beschriftet. Der Nummer wird hierbei das Präfix Tabelle oder Tab. vorangestellt. Aus den Referenznummern und der Beschreibung wird das am Anfang der Ausarbeitung stehende Tabellenverzeichnis erstellt. In Microsoft Word 2013 wird dies wiederum durch die Verwendung der Formatvorlage Beschriftung vereinfacht.
	Wenn bei einer Tabelle die Umrandung nicht sichtbar ist, so sollte der Einzug auf der linken Seite auf 0 cm gesetzt werden, damit der Text der Tabelle bündig zum Zeilenbeginn aller weiteren Textelemente steht.
	3.12	Abkürzungen
	Abkürzungen dürfen nur verwendet werden, wenn dies bei sehr häufiger Verwendung umfänglicher Begriffe zu einer erheblichen Ersparnis des Textumfanges führt und die Verständlichkeit des Textes nicht verschlechtert wird. Die Verwendung von allgemein gebräuchlichen Abkürzungen ist ebenfalls möglich. Ganz verzichten sollte man jedoch auf die Verwendung von selbst erfundenen Abkürzungen (Theisen, 2013, S. 213).
	Abkürzungen sind innerhalb des Textes bei der ersten Verwendung in runden Klammern nach der vollständigen Angabe des nicht abgekürzten Textes zu benennen und im Abkürzungsverzeichnis aufzuführen. Wird zum Beispiel vom Medizinproduktegesetz (MPG) gesprochen, so kann dies später nur noch als MPG bezeichnet werden.
	3.13	Zitate
	In der Seminararbeit werden die Hinweise auf die verwendete Literatur im Stil APA 6th durchgeführt. Dieser Stil wird von MS Word direkt unterstützt.
	3.14	Vorgabe für das Ausdrucken von Seiten
	Der Ausdruck erfolgt vorzugsweise beidseitig (Papier einsparen!). Bei der Formatierung wird nicht zwischen linker und rechter Seite unterschieden.
	Die Ausarbeitung muss als pdf-Datei abgegeben werden.

\fi

Vergleiche \href{https://git.beimgraben.net/frederik/SAT-WiSe-25-26/src/branch/main/HSRTReport}{HSRTReport.cls}

\section{Seitenformat}

\begin{itemize}
	\item DinA4, hochkant
\end{itemize}

\section{Seitenränder}

\begin{itemize}
	\item Linker Rand: 2 cm
	\item Rechter Rand: 2 cm
	\item Oberer Rand: 2 cm
	\item Unterer Rand: 2 cm
\end{itemize}

Jeder neue Absatz hat zum vorhergehenden Absatz einen Abstand von 6 Punkten. Für Listen (siehe letzter Absatz) gibt es die Formatvorlage „Listenabsatz“, bei der kein Abstand zwischen den Absätzen enthalten ist.
Die Ausrichtung des Textes und aller Überschriften erfolgt linksbündig mit Flattersatz. Nach Möglichkeit muss die Silbentrennung eingesetzt werden, damit der Flattersatz den Gesamteindruck der Druckseite nicht zu sehr beeinträchtigt. Es wird empfohlen, zur Silbentrennung sogenannte bedingte Trennstriche einzusetzen.

\section{Titel und Überschriften}

Titel und Überschriften müssen kurz und aussagekräftig sein. Überschriften dürfen sich nicht wiederholen, sondern müssen eindeutig voneinander zu unterscheiden sein.
Die Gliederung des Textes sollte maximal über 2 Ebenen erfolgen. Es ist darauf zu achten, dass nicht jeder Absatz eine eigene Kapitelüberschrift erhält (siehe oben).

% 3.4	Textabsätze
\section{Textabsätze}

Ein Absatz darf nie mit einer einzelnen Zeile auf einer neuen Seite enden oder mit einer einzelnen Zeile am Seitenende beginnen. Dies gilt in gleicher Weise für Überschriften.
Die Länge eines Absatzes ist auf maximal 400 Zeichen zu beschränken.
Absätze werden nicht willkürlich gesetzt. Sie sollen dazu dienen das Dokument nicht nur formal, sondern auch inhaltlich zu gliedern und somit das Lesen und Verstehen des Textes erleichtern.
Die Silbentrennung ist so einzusetzen, dass jede Zeile möglichst bis zum Zeilenrand beschrieben ist. Beim Einsatz von Blocksatz ist darauf zu achten, dass die Lücken zwischen den Worten nicht zu groß werden.
Zwischen den einzelnen Absätzen ist ein Abstand von 6 Punkten einzuhalten.

\section{Kopfzeile}
Der Aufbau der Kopfzeile erfolgt, mit Ausnahme der Titelseite, gemäß dem in diesem Dokument enthaltenen Beispiel und beinhaltet auch die Kapitelnummer und –überschrift der ersten Ebene.

\section{Fußzeile}
Der Aufbau der Fußzeile erfolgt, mit Ausnahme der Titelseite, gemäß dem in diesem Dokument enthaltenen Beispiel.

\section{Seitennummerierung}
Die Seitennummerierung erfolgt entsprechend der Darstellung in diesem Dokument. Die Zählung beginnt bei 0 für das Titelblatt, so dass die erste Verzeichnisseite die Nummer 1 aufweist.

\section{Fußnoten}
Fußnoten werden im Text durch eine hochgestellte Ziffer grundsätzlich nach einem Satzzeichen (Komma, Strichpunkt, Punkt, Fragezeichen, Ausrufungszeichen, Gedankenstrich) referenziert. Die Fußnotennummerierung erfolgt fortlaufend über die gesamte Arbeit.
Erläuterungen zu den Fußnoten erfolgen oberhalb der Fußzeile und reduzieren die Zeilenzahl des Textkörpers. Der Beginn einer Fußnote wird durch einen linksbündigen horizontalen Strich, dessen Länge ca. 30% Breite des Textblockes beträgt, vom Textkörper abgetrennt.
Die Fußnotentexte sollten nicht länger als 4 Zeilen sein. Ebenso sollte vermieden werden, den Umbruch der Fußnoten auf die nächste Seite zu erzwingen.

\section{Formeln}

% Polstellenberechnung – Formel 1 : Polstellenberechnung
\begin{equation}
	x = \frac{-b \pm \sqrt{b^2 - 4ac}}{2a}
\end{equation}
\myequations{Polstellenberechnung}
