%!TEX root = ../Main.tex
\makeglossaries

\newcolumntype{L}[1]{>{\raggedright\let\newline\\\arraybackslash\hspace{0pt}}p{#1}}
\newcolumntype{C}[1]{>{\centering\let\newline\\\arraybackslash\hspace{0pt}}p{#1}}
\newcolumntype{R}[1]{>{\raggedleft\let\newline\\\arraybackslash\hspace{0pt}}p{#1}}

\newglossarystyle{manualfixedwidth}{
	\setglossarystyle{long3colheader} % Use existing stable base style
	%\newcolumntype{R}{>{\raggedleft}p{2cm}}
	%\newcolumntype{W}{>{\raggedright}p{4.5cm}}
	%\newcolumntype{D}{>{\raggedright}p{9.5cm}}
	%\setlength\LTleft{-5pt}
	\renewenvironment{theglossary}
	{\begin{longtable}{
				@{}
				L{0.3\textwidth-\tabcolsep}
				p{0.59\textwidth-\tabcolsep}
				R{0.1\textwidth-\tabcolsep}
				@{}
			}}
			{\end{longtable}}
	\renewcommand{\arraystretch}{1.1}
}

\setglossarystyle{manualfixedwidth}

\renewcommand*{\entryname}{Wort/Abkürzung}
\renewcommand*{\descriptionname}{Bedeutung}
\renewcommand*{\pagelistname}{Seite(n)}
\glsenablehyper
\renewcommand*{\glsclearpage}{}
\renewcommand{\acronymname}{Abkürzungsverzeichnis}

%%% https://golatex.de/viewtopic.php?t=23348
% masculine genitive
\glsaddkey
{genitive}% key
{}% default value
{\glsentrygenitive}% no link cs
{\Glsentrygenitive}% no link ucfirst cs
{\glsgen}% link cs
{\Glsgen}% link ucfirst cs
{\GLSgen}% link all caps cs

% dative
\glsaddkey
{dative}% key
{}% default value
{\glsentrydative}% no link cs
{\Glsentrydative}% no link ucfirst cs
{\glsdative}% link cs
{\Glsdative}% link ucfirst cs
{\GLSdative}% link all caps cs
