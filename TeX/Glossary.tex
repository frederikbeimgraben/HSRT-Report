%!TEX root = ../Main.tex
\makeglossaries
\iffalse
	\newglossarystyle{manualfixedwidth}{
	\setglossarystyle{long3colheader} % Use existing stable base style
	\newcolumntype{R}{>{\raggedleft\arraybackslash}p{2cm}}
	\setlength\LTleft{-5pt}
	\renewenvironment{theglossary}%
	{\begin{longtable}{p{4cm}p{10cm}R}} % override only the tabular environment specifying widths
			{\end{longtable}}%
	}
\fi

\newglossarystyle{manualfixedwidth}{
\setglossarystyle{long3colheader} % Use existing stable base style
\newcolumntype{R}{>{\raggedleft\arraybackslash}p{2cm}}
\setlength\LTleft{-5pt}
\renewenvironment{theglossary}%
{\begin{longtable}{p{3cm}p{11cm}R}} % override only the tabular environment specifying widths
		{\end{longtable}}%
\renewcommand{\arraystretch}{1.5} % 1.5 times normal line height
}

\setglossarystyle{manualfixedwidth}

%\setglossarystyle{long3colheader}
\renewcommand*{\entryname}{Wort/Abkürzung}
\renewcommand*{\descriptionname}{Beschreibung}
\renewcommand*{\pagelistname}{Seite(n)}
\setlength{\glsdescwidth}{0.75\textwidth}
\glsenablehyper
\renewcommand*{\glsclearpage}{}
\renewcommand{\acronymname}{Abkürzungsverzeichnis}

%%% https://golatex.de/viewtopic.php?t=23348
% masculine genitive
\glsaddkey
{genitive}% key
{}% default value
{\glsentrygenitive}% no link cs
{\Glsentrygenitive}% no link ucfirst cs
{\glsgen}% link cs
{\Glsgen}% link ucfirst cs
{\GLSgen}% link all caps cs

% ====== CONTENT ======

\newglossaryentry{abstract}
{
	name=Abstract,
	description={Das Abstract beschreibt in wenigen Sätzen die Zielsetzung und das Ergebnis der Ausarbeitung. Das Abstract muss sich vollständig auf der Titelseite befinden. Die Zeichensatzformatierung wird in einem eigenen Absatz beschrieben  Das Abstract soll es den Lesern:innen ermöglichen, innerhalb von wenigen Augenblicken zu erfassen, welcher Inhalt hinter der Überschrift steckt und ob das Thema, aus Sicht der Leser:innen, zur weiteren Bearbeitung lohnt. Das Abstract ist keine verbale Beschreibung des Inhaltsverzeichnisses, sondern gibt kurz und knapp z.B. die Zielsetzung (z.B. Hypothese), die eingesetzten Methoden und die erzielten Ergebnisse / Erkenntnisse bekannt.}
}

\newacronym{a:tES}{tES}{Transkranielle Elektrostimulation}

\newglossaryentry{tES}
{
	name=Transkranielle Elektrostimulation,
	description={
			Die transkranielle Elektrostimulation, kurz \acrshort{a:tES}, ist ein nicht-invasives, neurostimulatorisches Verfahren, das schwache elektrische Ströme nutzt, um die neuronale Aktivität im Gehirn zu modulieren.
		},
	genitive=Transkraniellen Elektrostimulation
}

\newacronym{a:tDCS}{tDCS}{Transkranielle Gleichstromstimulation}

\newglossaryentry{tDCS}
{
	name=Transkranielle Gleichstromstimulation,
	description={
			Die transkranielle Gleichstromstimulation, kurz \acrshort{a:tDCS}, ist eine variante der \glsgen{tES}, die Gleichströme verwendet.
		}
}

\newacronym{a:tACS}{tACS}{Transkranielle Alternierende Stromstimulation}

\newglossaryentry{tACS}
{
	name=Transkranielle Alternierende Stromstimulation,
	description={
			Die transkranielle Alternierende Stromstimulation, kurz \acrshort{a:tACS}, ist eine variante der \glsgen{tES}, die Wechselströme verwendet.
		}
}
