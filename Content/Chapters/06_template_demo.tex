% !TEX root = ../../Main.tex
% ==============================================================================
% Kapitel 6: Template-Features Demonstration
% ==============================================================================
% Description: Demonstration aller Features des HSRTReport Templates
% ==============================================================================

\chapter{Template-Features Demonstration}
\label{chap:template_demo}

Dieses Kapitel demonstriert die verschiedenen Funktionen und Möglichkeiten des HSRTReport-Templates für wissenschaftliche Arbeiten.

\section{Textformatierung}
\label{sec:textformatierung_demo}

\subsection*{Grundlegende Textauszeichnungen}
\label{subsec:textauszeichnungen}

Das Template unterstützt verschiedene Textauszeichnungen:
\begin{listenabsatz}
	\item \textbf{Fetter Text} für wichtige Begriffe
	\item \emph{Kursiver Text} für Betonungen und Eigennamen
	\item \texttt{Maschinenschrift} für Code und Befehle
	\item \textsf{Serifenlose Schrift} für spezielle Hervorhebungen
	\item \textsc{Kapitälchen} für besondere Formatierungen
\end{listenabsatz}

\subsection*{Spezielle Zeichen und Symbole}
\label{subsec:spezielle_zeichen}

Das Template unterstützt korrekte Typografie:
\begin{listenabsatz}
	\item Gedankenstriche -- so wie hier -- für Einschübe
	\item Anführungszeichen: "`deutsche Anführungszeichen"' und ``englische Quotes''
	\item Auslassungspunkte \ldots{} mit korrektem Spacing
	\item Geschützte Leerzeichen: z.\,B. oder 10\,cm oder S.\,42
	\item Mathematische Symbole: $\alpha$, $\beta$, $\gamma$, $\sum$, $\int$
\end{listenabsatz}

\section{Listen und Aufzählungen}
\label{sec:listen_demo}

\subsection*{Verschiedene Listentypen}
\label{subsec:listentypen}

Standard-Aufzählung:
\begin{itemize}
	\item Erster Punkt
	\item Zweiter Punkt
	\item Dritter Punkt mit Unterpunkten:
	      \begin{itemize}
		      \item Unterpunkt A
		      \item Unterpunkt B
	      \end{itemize}
\end{itemize}

Nummerierte Liste:
\begin{enumerate}
	\item Erster Schritt
	\item Zweiter Schritt
	\item Dritter Schritt
\end{enumerate}

Kompakte Liste ohne Abstände (listenabsatz):
\begin{listenabsatz}
	\item Element ohne Abstand darüber
	\item Element ohne Abstand dazwischen
	\item Element ohne Abstand darunter
\end{listenabsatz}

Beschreibungsliste:
\begin{description}
	\item[Begriff 1] Erklärung des ersten Begriffs
	\item[Begriff 2] Erklärung des zweiten Begriffs
	\item[Begriff 3] Erklärung des dritten Begriffs
\end{description}

\section{Mathematische Formeln}
\label{sec:formeln_demo}

\subsection*{Inline-Formeln}
\label{subsec:inline_formeln}

Formeln können direkt im Text verwendet werden, wie z.\,B. $E = mc^2$ oder $a^2 + b^2 = c^2$. Auch komplexere Ausdrücke wie $\int_{0}^{\infty} e^{-x^2} dx = \frac{\sqrt{\pi}}{2}$ sind möglich.

\subsection*{Abgesetzte Formeln}
\label{subsec:abgesetzte_formeln}

Einfache nummerierte Gleichung:

\begin{equation}
	\nabla \times \vec{E} = -\frac{\partial \vec{B}}{\partial t}
	\label{eq:maxwell1}
\end{equation}
\myequations{Erste Maxwell-Gleichung (Faradaysches Induktionsgesetz)}

Mehrzeilige Gleichung mit Alignment:

\begin{align*}
	f(x) & = x^2 + 2x + 1        \\
	     & = (x + 1)^2 \numbereq
	\label{eq:binomial}
\end{align*}
\myequations{Binomische Formel}

Matrix-Darstellung:

\begin{equation}
	\mathbf{A} = \begin{pmatrix}
		a_{11} & a_{12} & a_{13} \\
		a_{21} & a_{22} & a_{23} \\
		a_{31} & a_{32} & a_{33}
	\end{pmatrix}
	\label{eq:matrix}
\end{equation}
\myequations{3x3-Matrix}

\section{Abbildungen und Grafiken}
\label{sec:abbildungen_demo}

\subsection*{Einfache Abbildung}
\label{subsec:einfache_abbildung}

\begin{figure}[h]
	\centering
	\fbox{\parbox{0.7\textwidth}{
			\centering
			\vspace{4cm}
			\Large Platzhalter für Abbildung\\
			\normalsize (Hier könnte eine Grafik, ein Diagramm oder ein Foto stehen)
			\vspace{4cm}
		}}
	\caption{Demonstrationsabbildung mit Platzhalter}
	\label{fig:demo_abbildung}
\end{figure}

Die Abbildung~\ref{fig:demo_abbildung} zeigt einen Platzhalter für eine echte Grafik. Abbildungen werden automatisch nummeriert und im Abbildungsverzeichnis aufgeführt.

\subsection*{Subfigures}
\label{subsec:subfigures}

Mit dem \texttt{subcaption}-Paket können mehrere Abbildungen nebeneinander platziert werden:

\begin{figure}[h]
	\centering
	\begin{subfigure}[b]{0.45\textwidth}
		\centering
		\fbox{\parbox{0.9\textwidth}{\centering\vspace{2cm}Bild A\vspace{2cm}}}
		\caption{Erste Teilabbildung}
		\label{fig:sub1}
	\end{subfigure}
	\hfill
	\begin{subfigure}[b]{0.45\textwidth}
		\centering
		\fbox{\parbox{0.9\textwidth}{\centering\vspace{2cm}Bild B\vspace{2cm}}}
		\caption{Zweite Teilabbildung}
		\label{fig:sub2}
	\end{subfigure}
	\caption{Zwei Abbildungen nebeneinander}
	\label{fig:subfigures_demo}
\end{figure}

\section{Tabellen}
\label{sec:tabellen_demo}

\subsection*{Einfache Tabelle}
\label{subsec:einfache_tabelle}

\begin{table}[h]
	\centering
	\caption{Messwerte-Demonstration}
	\label{tab:messwerte}
	\begin{tabular}{|l|c|c|c|c|}
		\hline
		\textbf{Messung} & \textbf{Zeit [s]} & \textbf{Spannung [V]} & \textbf{Strom [A]} & \textbf{Leistung [W]} \\
		\hline
		1                & 0                 & 12.0                  & 0.5                & 6.0                   \\
		2                & 10                & 11.8                  & 0.6                & 7.1                   \\
		3                & 20                & 11.5                  & 0.7                & 8.1                   \\
		4                & 30                & 11.2                  & 0.8                & 9.0                   \\
		\hline
	\end{tabular}
\end{table}

\subsection*{Komplexe Tabelle mit multirow und multicolumn}
\label{subsec:komplexe_tabelle}

\begin{table}[h]
	\centering
	\caption{Komplexe Tabellenstruktur}
	\label{tab:komplex}
	\begin{tabular}{|l|c|c|c|}
		\hline
		\multirow{2}{*}{\textbf{Kategorie}} & \multicolumn{3}{c|}{\textbf{Messwerte}}                                  \\
		\cline{2-4}
		                                    & \textbf{Min}                            & \textbf{Max} & \textbf{Mittel} \\
		\hline
		Temperatur [°C]                     & 18.5                                    & 24.3         & 21.2            \\
		Luftfeuchtigkeit [\%]               & 45                                      & 62           & 53              \\
		Druck [hPa]                         & 1013                                    & 1021         & 1017            \\
		\hline
	\end{tabular}
\end{table}

\section{Code-Listings}
\label{sec:code_listings}

\subsection*{Python-Code}
\label{subsec:python_code}

%TC:Ignore
\begin{lstlisting}[caption={Python-Beispiel: Fakultätsfunktion},label={lst:python_factorial},language=Python]
def factorial(n):
    """Berechnet die Fakultät einer Zahl rekursiv."""
    if n <= 1:
        return 1
    else:
        return n * factorial(n - 1)

# Verwendungsbeispiel
for i in range(10):
    print(f"{i}! = {factorial(i)}")
\end{lstlisting}
%TC:EndIgnore

\subsection*{LaTeX-Code}
\label{subsec:latex_code}

%TC:Ignore
\begin{lstlisting}[caption={LaTeX-Beispiel: Dokumentstruktur},label={lst:latex_example},language=TeX]
\documentclass{article}
\usepackage[ngerman]{babel}
\usepackage{amsmath}

\begin{document}
\section{Überschrift}
Dies ist ein Beispieltext mit einer Formel: $x^2 + y^2 = r^2$

\begin{equation}
    \int_{-\infty}^{\infty} e^{-x^2} dx = \sqrt{\pi}
\end{equation}
\end{document}
\end{lstlisting}
%TC:EndIgnore

\section{Querverweise und Zitationen}
\label{sec:querverweise}

\subsection*{Interne Querverweise}
\label{subsec:querverweise_intern}

Das Template unterstützt intelligente Querverweise mit dem \texttt{cleveref}-Paket:
\begin{listenabsatz}
	\item Verweis auf Kapitel: siehe \cref{chap:template_demo}
	\item Verweis auf Abschnitt: siehe \cref{sec:formeln_demo}
	\item Verweis auf Gleichung: siehe \cref{eq:maxwell1}
	\item Verweis auf Abbildung: siehe \cref{fig:demo_abbildung}
	\item Verweis auf Tabelle: siehe \cref{tab:messwerte}
	\item Verweis auf Listing: siehe \cref{lst:python_factorial}
\end{listenabsatz}

\subsection*{Fußnoten}
\label{subsec:fussnoten_demo}

Fußnoten\footnote{Dies ist eine Beispiel-Fußnote mit zusätzlichen Informationen.} können für ergänzende Informationen verwendet werden. Sie sollten jedoch sparsam eingesetzt werden\footnote{Eine zweite Fußnote zur Demonstration der automatischen Nummerierung.} und nicht länger als vier Zeilen sein.

\section{Glossar und Abkürzungen}
\label{sec:glossar_demo}

\subsection*{Verwendung von Glossareinträgen}
\label{subsec:glossar_verwendung}

Das Template nutzt das \texttt{glossaries}-Paket für die Verwaltung von Fachbegriffen und Abkürzungen:
\begin{listenabsatz}
	\item Erster Aufruf eines Glossarbegriffs: \gls{Textkörper}
	\item Zweiter Aufruf desselben Begriffs: \gls{Textkörper}
	\item Verwendung einer Abkürzung: \gls{MPG}
	\item Nochmalige Verwendung: \gls{MPG}
\end{listenabsatz}

Die Begriffe werden automatisch in das entsprechende Verzeichnis aufgenommen.

\section{Spezielle Umgebungen}
\label{sec:spezielle_umgebungen}

\subsection*{Theorem-ähnliche Umgebungen}
\label{subsec:theoreme}

Obwohl nicht standardmäßig aktiviert, können bei Bedarf Theorem-Umgebungen definiert werden für:
\begin{listenabsatz}
	\item Definitionen
	\item Sätze (Theoreme)
	\item Lemmata
	\item Korollare
	\item Beweise
	\item Beispiele
\end{listenabsatz}

\subsection*{Infoboxen und Warnungen}
\label{subsec:infoboxen}

Das Template kann erweitert werden um spezielle Boxen für:
\begin{listenabsatz}
	\item Wichtige Hinweise
	\item Warnungen
	\item Tipps und Tricks
	\item Zusammenfassungen
\end{listenabsatz}

\section{Erweiterte Features}
\label{sec:erweiterte_features}

\subsection*{TikZ-Grafiken}
\label{subsec:tikz}

Mit TikZ können komplexe Diagramme direkt in \LaTeX{} erstellt werden. Das Template lädt die notwendigen Pakete automatisch.

\subsection*{SVG-Integration}
\label{subsec:svg}

SVG-Dateien können direkt eingebunden werden mit dem \texttt{\textbackslash includesvg}-Befehl, was besonders für Vektorgrafiken aus Inkscape nützlich ist.

\subsection*{Hyperlinks}
\label{subsec:hyperlinks}

Das Template unterstützt:
\begin{listenabsatz}
	\item Automatische Verlinkung von Querverweisen
	\item Klickbare URLs: \url{https://www.example.com}
	\item E-Mail-Links: \href{mailto:example@domain.com}{example@domain.com}
	\item Verlinktes Inhaltsverzeichnis im PDF
\end{listenabsatz}

\section{Zusammenfassung der Template-Features}
\label{sec:template_zusammenfassung}

Das HSRTReport-Template bietet eine umfassende Lösung für wissenschaftliche Arbeiten mit:
\begin{listenabsatz}
	\item Automatischer Formatierung gemäß den Vorgaben
	\item Konsistenter Typografie und Layout
	\item Intelligenter Verwaltung von Verzeichnissen
	\item Flexiblen Möglichkeiten für Formeln, Abbildungen und Tabellen
	\item Professioneller PDF-Ausgabe mit Hyperlinks
	\item Einfacher Anpassbarkeit über Konfigurationsdateien
\end{listenabsatz}

Alle diese Features tragen dazu bei, dass sich der Autor auf den Inhalt konzentrieren kann, während das Template die korrekte Formatierung übernimmt.

% ==============================================================================
% End of Kapitel 6
% ==============================================================================
